\documentclass[a4paper,10pt]{article}
\usepackage[utf8]{inputenc}


\newcommand{\sect}[1]{Section ~\ref{sect:#1}}
%opening
\title{Sprint 3 Summary}

\parindent 0pt
\parskip 12pt

\begin{document}

\maketitle

\section{Backend Team}

The overall goal for the third sprint was to have a solid distributed Erlang NetInf
architecture that can handle storing and searching of content. In section 1.1 we
describe the goals set for sprint 3 by the back-end team, section 1.2 provides a
brief summary of the functionalities implemented in this sprint and in section
1.3 we describe some ideas that we have for the next sprint.


\subsection {Goals}
\label{sect:goals}
For the third sprint we had the following goals:
\begin{enumerate}
 \item Store Content
 \item Database storage
 \item Refactor Code
 \item Search based on meta data
 \item Message forwarding in a distributed NetInf system
\end{enumerate}
\subsection {Functionalities Implemented}
In this section we briefly describe each functionality that we implemented during
the third sprint.
\subsubsection {Store Content}
The back-end application is able to handle a ’full Put’ which means we can
store the bit level objects contained inside the published NDOs in the node that
receives the pulish request.
\subsubsection {Database storage}
The back-end application is able to store the published NDO’ inside a database.
We implemented Riak to store information related to the NDO’s. Riak is an
open source database that stores data using a simple key/value scheme.
\subsubsection {Search based on meta data}
The back-end appliction is able to search NDOs based on meta data. The search
functionality will return a list of NDO names that match the search tokens sent
by the front end application.
\subsubsection {Message forwarding in a distributed NetInf system}
The back-end application is able to forward messages to other NetInf nodes
to get and search NDOs. We implemented simple service discovery protocol
(SSDP) which uses UDP as the underlying protocol to broadcast IP addresses
on a network. This way each node can broadcast its IP address on the network
and other nodes will keep a list of these IP addresses for the purpose of message
forwarding to other nodes.

\subsection {Sprint 4 Ideas}
We have following ideas for the next sprint but we will decide at a later stage
which one of these to implement:
\begin{itemize}
\item Support a signature-based naming scheme, to allow dynamic NDOs
\item Implement some kind of access control mechanism on the published NDOs
\item Compare our Erlang implementation of Open NetInf with other imple-
mentations based on some measurement criteria and collect results
\item Truncate hash algorithm
\item Implement another database for storing NDO’s e.g. SQLite
\item Make our application work with support for bluetooth on laptops
\end{itemize}
\section{Frontend Team}

The third sprint will be demoed on Friday, the 23rd of November.
Our goals are outlined in \sect{goal}. For a thorough understanding of
what has to be accomplished, we will explain our tasks corresponding to our
goals more deeper in \sect{explanation}. \sect{demo} describes how
we plan to demo our progress and \sect{sprint4} includes ideas for the upcoming sprint.

\subsection{Goal}
\label{sect:goal}
For the third sprint we have the following main goals:
\begin{itemize}
 \item Application-wise: The application should be able to $\dots$
 \begin{enumerate}
  \item search for a simple webpage through a url
  \item publish and request content from the local database and the NetInf node (NRS)
  \item show the requested webpage within the WebView
 \end{enumerate}
 \item Test and clean up the current code and repository
\end{itemize}

\subsection{Tasks}
\label{sect:explanation}
Corresponding to the goals, the following sections will give more information
about our actual tasks.

\subsubsection{Application}
\subsubsection{Search}
As soon as the user types in a url, the application will search for the
corresponding information object. The search request can be sent to the local database
as well as to another NetInf node (NRS). Right now, the NRS and the database don't have
the same interface for searching, but for setting and getting information objects (see \sect{request}).

\subsubsection{Publish \& Request}
\label{sect:request}
A request will be send to all existing Resolution Services. In this sprint we will 
add a local resolution service (LRS) that represents the interface to the local database.
Therefore the database will have the same interface as the NRS for publishing and
retrieving information objects.

Since the backend team will provide cashing functionalities in this sprint, our application will
have the possibility to publish to and retrieve from both NRS and LRS.

\subsubsection{Show content}
The view of our application shown before as a prototype will be created. The necessary functionalities
need to be implemented and the WebView will be used for displaying simple webpages containing only text.

Accordingly, the user will be able to request a simple webpage and view it within the application.

\subsubsection{Clean-Up}
Until the demo we want to have refactored and tested code. All exceptions 
should be handled in a proper way. Jenkins should be set up correctly in order to provide continuous integration.
Reviews will be done mainly by pair programming. 
We want to introduce Util classes for common functionality. 
The main properties will be read from a properties file.

\subsection{Demo}
\label{sect:demo}
In order to demo our current progress we plan to do the following:

\subsubsection{Search demo}
Type in a web url and press "GO".

\textbf{Case 1.: Fetching from uplink}

 \begin{enumerate}
  \item The application doesn't have the IO
  \item It triggers a search request to the NRS
  \item The NRS does not have the content either and does not have a locator list
  \item The application fetches the webpage from the uplink
  \item The application displays the content of a simple web page
  \item The application publishes the IO to NRS and LRS
 \end{enumerate}
 
 \textbf{Case 2.: Fetching from LRS}
 
 \begin{enumerate}
  \item The application has the IO
  \item The application displays the content of a simple web page
 \end{enumerate}

\textbf{Case 3.: Fetching from NRS}

 \begin{enumerate}
  \item The application doesn't have the IO
  \item It triggers a search request to the NRS
  \item The NRS has the IO and maybe a locator list
  \item The application fetches the webpage from the NRS
  \item The application displays the content of a simple web page
  \item The application informs the NRS about being a locator and publishes the IO the the LRS
 \end{enumerate}

\textbf{Case 4.: Fetching from LRS}
\begin{enumerate}
  \item The application doesn't have the IO
  \item It triggers a search request to the NRS
  \item The NRS has a locator list
  \item The application fetches the webpage from another device
  \item The application displays the content of a simple web page
  \item The application informs the NRS about being a locator and publishes the IO the the LRS
 \end{enumerate}

\subsection{Sprint 4 Ideas}
\label{sect:sprint4}
Our current ideas for the next sprint are:

\begin{itemize}
 \item Handling real web pages 
 \item Adjusting the interfaces of LRS and NRS to completely be the same
 \item Using the Search Controller of OpenNetInf
\end{itemize}

\end{document}
