\documentclass[12pt]{article}
\usepackage{listings}
\newcommand{\piRsquare}{\pi r^2}		% This is my own macro !!!

\title{Extreme Programming}			% used by \maketitle
\author{Jon Borglund \\ Faroogh Hassan}		% used by \maketitle
\date{October 5, 2012}					% used by \maketitle
\begin{document}
\maketitle					% automatic title!

\section{Purpose}
This document is a brief introduction to some of the Extreme Programming (XP) methodologies available.

\section{Values}
The four values of XP are: 
\begin{itemize}
\item Communication 
\item Simplicity 
\item Feedback 
\item Courage 
\end{itemize}


\section{Fundamental principles}
\begin{description}
\item[Rapid feedback] Get feedback, interpret it, and put what is learned back into the system as quickly as possible. 
\item[Assume simplicity] Solve today's job today and trust your ability to add complexity in the future where you need it.
\item[Incremental change] Solve a problem with a series of small changes.
\item[Embracing change] 
\item[Quality work] The only possible values are "excellent" and insanely "excellent". 
\end{description}

\section{Practices}
\paragraph{The Planning Game} Quickly determine the scope of the next release by 
combining business priorities and technical estimates. As reality overtakes the 
plan, update the plan. 

\paragraph{Small releases} Put a simple system into production quickly, then release new 
versions on a very short cycle. 

\paragraph{Metaphor} Guide all development with a simple shared story of how the 
whole system works. 

\paragraph{Simple design} The system should be designed as simply as possible at any 
given moment. Extra complexity is removed as soon as it is discovered. 

\paragraph{Testing} Programmers continually write unit tests, which must run flawlessly 
for development to continue. Customers write tests demonstrating that 
features are finished. 

\paragraph{Re-factoring} Programmers restructure the system without changing its  behaviour to remove duplication, improve communication, simplify, or add flexibility. 

\paragraph{Pair programming} A programming task is solved by two programmers at one machine. One is in the control of the computer and the other reviews the code and gives feedback. The roles are changed in periods of hours or minutes. The pairs should be mixed randomly within the team.   

\paragraph{Collective ownership} Anyone can change any code anywhere in the system at any time. 

\paragraph{Continuous integration} Integrate and build the system many times a day, every time a task is completed. 

\paragraph{40 hour week} Work no more than 40 hours a week as a rule. Never work overtime a second week in a row. 

\paragraph{On-site customer} Include a real, live user on the team, available full-time to 
answer questions. 

\paragraph{Coding standards} Programmers write all code in accordance with rules 
emphasizing communication through the code. 

\end{document}             % End of document

