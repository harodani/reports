\section{Team Building}
In order to maintain a friendly environment at the work place we undertook certain measures which are explained below:

\subsection{Fika}
Fika is a Swedish tradition where people take a break to drink coffee or tea. Every Wednesday one of the group members would bring something that we could eat with coffee or tea. The fika break started at 15.00 on Wednesdays and it lasted for a maximum of 30 minutes. We used this time to get to know each other better and discussed everything from politics to weather. Apart from the weekely fika we sometimes also had the 'late arrival punishment fika' where anyone who didn't show up on time would bring something for the whole team to eat at the fika break.  

\subsection{Birthdays}
Another team bulding measurment that we took was to celebrate birthdays of group members. For this purpose we collected money at the start of the project from every one. This 'birthday fund' was used to buy cake for the birthday celebration. We would then take a break from work to celebrate the birthday of the person and eat the cake with coffee or tea. 

\subsection{Eating out}
Every now and then we used to go to a restaurant or a student nation to eat something for lunch or dinner. Also when the weather was nice at the beginning of the project, we arranged an outdoor BBQ which gave the team members the opportunity to get to know each other.    