\section{Team Building}

In order to maintain a friendly environment at the work place and to make people
feel comfortable and confident with each other, we arranged a series of activities
which are explained below:

\subsection{Fika}
Fika is a Swedish tradition where people take a break to drink coffee or tea. Every Wednesday one
of each team would bring something to share with the rest of the group. The fika break started at 15.00 
on Wednesdays and it lasted for as long until we could regain our focus on the project. We used this time to get to know each other better and discussed 
everything from politics to weather. Apart from the weekely fika, we sometimes also had the 'late arrival punishment fika',
where anyone who didn't show up on time would arrange an extra fika.

\subsection{Birthdays}
We also used to celebrate birthdays of group members. For this purpose we collected 
money at the start of the project from everyone. This 'birthday fund' was used to buy a cake for the birthday celebration. 
We would then take a break from work to celebrate the birthday of the person and eat the cake with coffee or tea. 

\subsection{Eating out}
Every now and then we used to go to a restaurant or a student nation to eat something for lunch or dinner. Also when the weather 
was nice at the beginning of the project, we arranged an outdoor BBQ which gave the team members the opportunity to spend time with each
other outside of the working environment.