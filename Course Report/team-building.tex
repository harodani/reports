\section{Team Building}

In order to maintain a friendly environment at the work place and to make people
feel comfortable and confident with each other, the team arranged a series of activities which are explained in the upcoming sections.

\subsection{Fika}
Fika is a Swedish tradition where people take a break to drink coffee or tea. Every Wednesday two group members would bring something to share with the rest of the group. The fika break started at 15.00 
on Wednesdays. This time to get to know each other better and discussed 
everything from politics to weather. Apart from the weekely fika, there was also the 'late arrival punishment fika',
where anyone who didn't show up on time would arrange an extra fika.

\subsection{Birthdays}
Group members who had a birthday during the course were celebrated. For this purpose 
money was collected at the start of the project from everyone. This 'birthday fund' was used to buy a cake for the birthday celebration. 
The teams would then take a break from work to celebrate the birthday of the person and eat the cake with coffee or tea. 

\subsection{Eating out}
Every now and then the team used to go to a restaurant or a student nation to eat something for lunch or dinner. Also when the weather 
was nice at the beginning of the project, there was an outdoor BBQ which gave the team members the opportunity to spend time with each
other outside of the working environment.

\subsection{Bowling}
After the midcourse presentation at Ericsson Research in Kista, they invited us to bowl, drink beer and eat dinner at a bowling alley in Stockholm. This was a really good evening, since the students got a chance to mingle with the people involved in the project on a more personal level.
