\section{Introduction}
The main purpose of this project was to build an application based on the NetInf protocol. NetInf nodes in a NetInf network support the creation, exchange and storage of units of data that can be identified and accessed using a URI-based naming scheme. \cite{netinfspecs} Ericsson Research acted as the customer for this project whereas the IT department of Uppsala University facilitated us with a project room, software and hardware infrastructure. 

The project group was divided into two teams where one team was dedicated for developing the front-end of the 
application whereas the other team was given the task to develop the back-end. The front-end group was called 
LISA (abbreviation for "Look! I see ants!" which sounds like ICNs, information-centric networking) and consisted of five members whereas 
the back-end team was named ERNI (Erlang NetInf) and was made up of eight individuals. 

Starting off with the front-end's lose requirement to use NetInf within an application that can communicate with
the back-end's server, the front-end decided to develop a mobile browser called "Elephant". Elephant looks like a normal browser from an end-user's perspective. So what differentiates a normal browser from the developed one? Instead of using normal host-based networking, a more information-centric networking (ICN) approach is used. Many recent papers and research efforts have noted that we should move the Internet away from its current reliance on purely point-to-point primitives to designs that make the Internet more data-oriented or content-centric. \cite{ghodsietal}  

Therefore the browser is a proof-of-concept towards a new networking model where Information Objects are in focus.
For the development Ericsson provided Android phones. Thus, Java was the language of choice.

The back-end team developed the Name Resolution Service (NRS) to perform the back-end operations using Erlang/OTP. The details of the functionalities developed 
by the front-end and back-end teams can be found in the product report. 

Both teams followed Scrum as the software development methodology. You can find a short description about Scrum and
how we used Scrum in \sect{scrum}.

This document describes in detail the various tools, methodology and equipments we used to achieve our goals. 
