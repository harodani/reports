\section{Project Management}

There have been many suggestions from previous instances of this course to use project management/issue trackers. Initially both teams were using Redmine but then decided to diverge and try out another project management tool. 

\subsection{Redmine}

The ERNI team decided to use Redmine for the duration of the project. The tool provided useful features such as
\begin{itemize}
\item Wiki
\item Version Control explorer
\item Bug and Issue tracking
\item Time keeping
\item File storage
\end{itemize} 

Redmine is easy to install and free, including plenty of plugins for various needs. Originally both teams started the first sprint with Redmine installed and two projects configured. Redmine was used heavily for the first sprint by ERNI for creating issues and trying to keep track of time spent for each task in order to aid in planning the time required for each task. 

However by sprint two's retrospective and a changeover to a new SCRUM master, the ERNI team decided to drop the issue tracking and the time keeping as they felt it was too much overhead and did not help so much to estimate the time required on tasks. The wiki became the most extensively used feature of this tool followed by the Version control explorer. 

Instead of creating issues and assigning them the ERNI team opted for a more simpler solution of writing tasks on post it notes and added them to a scrum board. Even though this was simpler, it became messier as sprints became longer.

\subsection{JIRA}

The LISA team wanted to test out JIRA since they had many issues getting the plugins and the right functionality for their team. JIRA unlike Redmine is not free, and comes with a 3 month trial license. LISA felt that this provided much more functionality than Redmine and much more easily customizable.

JIRA worked well for LISA as each member was at one point SCRUM Master, and were able to keep work logs, issues and assign tasks easily. The major point for using JIRA was persistence as each scrum sprint was catalogued.