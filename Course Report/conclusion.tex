\section{Conclusion}
At the end of this project we can proudly say that this project was a success for us. We achieved almost all the goals that we had set for ourselves at the beginning of the project and during each sprint. Our customer Ericsson Research was also happy with our performance and they offered thesis opportunities to a number of our team members to continue working on the application that we built. 

\Item{\textbf{What went well:}}\\
Even though we think that we were not 100 percent efficient with Scrum but it helped us a lot during all stages of this project. The morning Scrum meeting kept everyone updated on the progress of the project. The sprint planning and retrospective meetings gave us the opportunity to participate in meaningful discussions regarding how to build different functionlaties as part of our appliation. We are thankful to the teaching staff for arranging different workhops for us. The Agile, Erlang and Testing workshops by experienced professionals who have been part of the industry for many years were very helpful. 

For those who are going to be working with Erlang in future instances of this course, Erlang OTP in action is a great resource and helped greatly in the beginning of the course. Traditionally the course has a 2 week erlang workshop, we however had 1 week self-study (Erlang OTP in action and the online resource learn you some erlang for great good - http://www.Learnyousomeerlangforgreatgood.com) followed by 2 day Erlang workshop. We believe this was a much better way to handle things since we got to make all the mistakes before and ask valuable questions when the expert came in. 

The weekly fika is also something we would like to recommend the future students to continue. It is a good way to keep a friendly environment in the team and to get to know each other.     

\Item{\textbf{What did not went well:}} 
Another thing that we learnt during the course of this project was that not everything is going to be the way we plan it to be. We had problems with estimating the duration of different tasks in almost all the sprints. At times we overestimated the time we assigned for the completion of a particular task and at times we underestiamted it. 

Another problem that we had was to choose the right software tool for a particular purpose. At the beginning of this project we spent too much time on installing and reading about the tools that we never used later because we found a better tool. So it is important to choose the right tool from the beginning to save time and resources. To give an example, we installed Buildbot for continuous integration but found it difficult to learn and manage so we switched to Jenkins instead. So our advice to future students is to spend some time in invstigating what is the best tool that is easy to use and can be learnt quickly.          

\Item{\textbf{What can be improved:}}
 At times durng this project we built functionalities that we had to scrap later because they were not within the scope of the project. It is important to have a discussion before starting to code anything and have the consensus of the team on everything so that you do not end up building something that would be thrown away later. Communication is very important and team members should not shy away from asking quiestions or demanding clarifications on anything.  
 
The communication with customer is also an important part of any software project and we think that for future instances of this course the customer should be involved more in the project and should provide concrete requirements. In our case even though Ericsson Research was the official customer but at most times we had to act as a customer for ourselves and make decisions that otherwise the real customer has to make. 
