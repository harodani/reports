\section{Conclusion}
At the end of this project both teams can proudly say that this project was a success. 
Almost all the goals that we had set for ourselves at the beginning of the project 
and during each sprint were achieved and those which were not were achieved later on. The client, Ericsson Research, was happy with the performance and 
offered thesis opportunities to a number of team members to continue working on the 
application that was built. 

\begin{itemize}
\item{\textbf{What went well:}}\\
Even though the teams were not 100\% efficient, Scrum helped a lot 
during all stages of this project. The daily stand-up meeting kept everyone updated on the 
progress of the project. The sprint planning gave everyone the opportunity 
to participate in meaningful discussions regarding how to build different functionalities into the product. A lot of feedback was gained during the retrospectives that made it possible
to improve the product development. The teams are thankful to the teaching staff for arranging different workhops for us. 
The Agile, Erlang and Testing workshops, held by experienced professionals who have been part of the 
industry for many years, were very helpful. 

For those who are going to be working with Erlang in future instances of this course, Erlang 
OTP in action is a great resource and helped in the beginning of the course. Traditionally 
the course has a two week Erlang workshop, but this instance had one week of self-study (Erlang OTP In Action and 
the online resource learn you some erlang for great good\footnote{http://www.Learnyousomeerlangforgreatgood.com}) 
followed by two days of an Erlang workshop. The teams believe this was a much better way to handle things since the opportunity to 
make all the mistakes before and ask valuable questions afterwards when the expert came was there. 

The weekly fika is also something that is recommended to future students. It is a 
good way to keep a friendly environment in the team and to get to know each other.     

\item{\textbf{What did not go well:}}\\ 
Another thing learnt during the course of this project was that not everything is going to be 
the way it is planned to be. There were problems with estimating the duration of different tasks in almost all 
the sprints. At times the teams overestimated the time assigned for the completion of a particular task and 
at times it was underestimated. 

Another problem was to choosing the right software tool for a particular purpose. At the 
beginning of this project a considerable amount of time was spent on installing and reading about the tools that were never 
used later because of better options that we did not know about in advance. To give an example, the team installed Buildbot for continuous integration but found 
it difficult to learn and manage so a switch to Jenkins was made instead. The advice to future students is to 
spend some time in investigating what is the best tool that is easy to use and can be learnt quickly.

\item{\textbf{What can be improved:}}\\
 At times during this project there were built functionalities that had to be scrapped later because they were not 
 within the scope of the project. It is important to have a discussion before starting to code anything 
 and have the consensus of the team on the overall design, but not on every implementation. Communication is very important and team members should not shy away from asking 
 questions or demanding clarifications on anything.  
 
The communication with client is also an important part of any software project and the teams think that 
for future instances of this course the client should be involved more in the project and should 
provide concrete requirements. In our case Ericsson Research was the official client 
but at most times the teams had to act as a client for themselves and make decisions that otherwise the real 
client would have to make. 
\end{itemize}

Furthermore there are notes to future students that take this course and guides for setting up and configurating tools.
Always make sure that a working agreement exists (timing, working hours, fika rules and much more) and
is kept by all team members! It is important to stick to what everyone has agreed on or else
conflicts might arise. 
Remember to commit to the team. Project CS can be a course which provides the chance to get to know new friends but it also comes with a big responsibility.
It is a great opportunity for an individual to work in such a big group since
it resembles  the work environment in the future.
