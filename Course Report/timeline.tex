\section {Timeline}

The following is a timeline of both team's scrum sprints. Please note that we were generally synchronized with each other until sprint 4 where ERNI chose to have two one week sprints instead of one two week sprint. The result was a series of one week sprints for the ERNI team ending with seven sprints in total, while LISA had only six.

The course ran over a period of four and a half months starting in September until January 18th, with two major presentations. This particular instance of the course was a unique one since our group did not have a very concrete project laid out in terms of requirements. Therefore our real work and Scrum sprints started slightly later than other course instances. Furthermore we were too few people in order to have two groups
with two completely different clients.

\subsection{Pre-Scrum}
During the month of September the group met with the client who presented the idea, although it was not very concrete at the time. The lack of tight requirements gave us the freedom to choose the direction of where the project was going to go, however many felt that it was a little too free resulting in many weeks of research and no real direction. 

In general the first few weeks of the course start with an Erlang workshop which consists of two weeks of intensive Erlang, however our project was so open that we had no idea on whether we would need Erlang at all.

We started our first sprint after approximately one month.

\subsection {LISA}
\framebox(360, 85){
	\minibox{
	\textbf{Sprint 1}\\
	\begin{tabular}{p{3cm}p{8cm}}
	Scrum Master & Thiago\\
	Goal & Send and receive a message to the backend's server\\
	Sprint length & 2 weeks\\
	Met & Yes
	\end{tabular}
	}
}

Sprint 1 started off with setting up our working environment.
This included reading up on and configuring the tools we chose to use during the project,
which are described in \sect{tools}.

We had a Skype meeting with Hugo Negrette and Miguel Sosa, who worked on a
similar project before. Since their Master Thesis was a fundamental starting point
for us, they helped us understanding critical parts of their code.
\todo{Cite the master thesis of Hugo and Miguel}

Finally, we implemented a simple application that could send a message to
the backend team's Name Resolution Service (NRS) and receive an answer. As
we had more time than we expected, we started implementing parts of
the Bluetooth communication.

\framebox(360, 100){
	\minibox{
	\textbf{Sprint 2}\\
	\begin{centering}
	\begin{tabular}{p{3cm}p{8cm}}
	Scrum Master & Thiago\\
	Goal & Communicate with another device using Bluetooth \& Publish and retrieve content with metadata\\
	Sprint length & 2 weeks\\
	Met & Yes
	\end{tabular}
	\end{centering}
	}
}

Now that we succeeded to communicate with the backend's NRS,
we needed to send messages according to the NetInf specifications.
Thus, we investigated the specifications and embedded those message regulations
into our code. 

We designed our first architecture draft based on OpenNetInf and implemented
the most important modules for sending/receiving messages to/from the NRS -
this time using OpenNetInf. We managed to share Information Objects between
Android phones using Bluetooth. 

At this stage, the application could request a content using a short hash.
The specified hash was sent to the NRS and received a list of locators, that own
the requested object, in return. The locators were Bluetooth MAC adresses.
The application was then starting a Bluetooth discovery in order to find out
whether one of these locators was within reach. If so, a Bluetooth communication
to that locator was established. The hash was sent to the connected device, which
replied with the file that was identified by the hash.


\todo{cite netinf specs}

\framebox(360, 98){
	\minibox{
	\textbf{Sprint 3}\\
	\begin{tabular}{p{3cm}p{8cm}}
	Scrum Master & Paolo\\
	Goal & Successful presentation. Search content, cache and retrieve content from the NRS, implement a minimal web interface\\
	Sprint length & 3 weeks\\
	Met & Yes
	\end{tabular}
	}
}

Within the first week we mainly worked on the presentation for our review at Ericsson in Kista.
We prepared a paper prototype (consisting of screen shots of the final application) of our 
project idea of creating a browser application that we later called
Elephant.
The application should look like a normal browser and behave as expected. The main difference
we wanted to achieve is that the browser uses information-based networking instead of location-based
networking. Our idea was to create a browser using NetInf in a way, that the user does not
need to know what actually happens. It would be a first step of changing our way of 
networking. \textit{Note:} Using the paper prototype helped us a lot to agree on our
project in the end. Our client could understand our ideas and was very happy to
see where the project was going.

Before continuing adding new functionalities, we took some time during the third sprint
to clean up and refactor the code. We restructured our git branches due to our
experience we gained from the previous sprints. The structure is described in \sect{git}.

The new functionalities we added at last were publishing (register the 
device as a locator), full-put (uploading the content to the NRS) and
searching for contents using URLs. Furthermore we added a Local Resoltion Service (LRS)
besides to the NRS. The LRS looked up a content in a local database, that we
designed and implemented within this sprint. 

Now our application could search for content within a local database and in a remote
NRS by a URL. The response was a corresponding hash, that identified a
web page associated to that URL. In case the NRS owned the web page that
was searched for, we directly retrieved that web page instead of a hash.
Using the hash, we could get a web page from the LRS or a list of locators
from the NRS. In case of a list of locators, the device would start to
connect to other devices and download the content through Bluetooth.
As soon as we retrieved the web page, we could register ourself as a 
locator in the NRS or even upload the content to the NRS.
At this stage we displayed simple HTML web pages within a WebView environment, without any pictures or
java scripts.


\framebox(360, 75){
	\minibox{
	\textbf{Sprint 4}\\
	\begin{tabular}{p{3cm}p{8cm}}
	Scrum Master & Kim-Anh\\
	Goal & Higher granularity browsing\\
	Sprint length & 2 weeks\\
	Met & Yes
	\end{tabular}
	}
}

As requested from Ericsson, we separated our application into two
applications: Elephant, our browser that uses the services of another application,
the NetInf services. 

In sprint 4 we created our final application. This included
creating the minimal browsing functionalities: Handling clicks on links
as well as displaying web pages correctly as they are displayed in
other browsers. The main challenge was to intercept the Android WebView
to redirect resource requests to our NetInf services, instead of simply
downloading them from uplink.

Furthermore we needed to add settings and help pages. The user should
be able to decide which NRS to connect to and whether she wants to upload content to the NRS or
register her device as a locator.

After sprint 4 we had our final application that offered a browsing functionality
based on NetInf. Since some bugs were left to be reviewed, we tackled these
in the upcoming sprint.


\framebox(360, 85){
	\minibox{
	\textbf{Sprint 5}\\
	\begin{tabular}{p{3cm}p{8cm}}
	Scrum Master & Linus\\
	Goal & Running application without bug (not necessarily with Bluetooth Convergence Layer)\\
	Sprint length & 2 weeks\\
	Met & Yes
	\end{tabular}
	}
}

\framebox(360, 80){
	\minibox{
	\textbf{Sprint 6}\\
	\begin{tabular}{p{3cm}p{8cm}}
	Scrum Master & Harold\\
	Goal & Sucessful presentation and a product and course report\\
	Sprint length & 2 weeks\\
	Met & Yes
	\end{tabular}
	}
}

Overall goal of the project:
Final product status:

\subsection {ERNI}

Sprint 1\\
Scrum Master:
Goal: 
Sprint length:
Met:

Sprint 2\\
Scrum Master:
Sprint length:
Goal: 
Sprint length:
Met:

Sprint 3\\
Scrum Master:
Goal: 
Sprint length:
Met:

Sprint 4\\
Scrum Master:
Goal: 
Sprint length:
Met:

Sprint 5\\
Scrum Master:
Goal: 
Met:
Sprint length:

Sprint 6\\
Scrum Master:
Goal: 
Sprint length:
Met:


Overall goal of the Project:
Final product status:
