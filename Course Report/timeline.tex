\section {Timeline}

The following is a timeline of both team's scrum sprints. Please note that we were generally synchronized with each other until sprint 4 where ERNI chose to have two one week sprints instead of one two week sprint. The result was a series of one week sprints for the ERNI team ending with seven sprints in total, while LISA had only six.

The course ran over a period of four and a half months starting in September until January 18th, with two major presentations. This particular instance of the course was a unique one since our group did not have a very concrete project laid out in terms of requirements. Therefore our real work and Scrum sprints started slightly later than other course instances. Furthermore we were too few people in order to have two groups
with two completely different clients.

\subsection{Pre-Scrum}
During the month of September the group met with the client who presented the idea, although it was not very concrete at the time. The lack of tight requirements gave us the freedom to choose the direction of where the project was going to go, however many felt that it was a little too free resulting in many weeks of research and no real direction. 

In general the first few weeks of the course start with an Erlang workshop which consists of two weeks of intensive Erlang, however our project was so open that we had no idea on whether we would need Erlang at all.

We started our first sprint after approximately one month.

\subsection {LISA}
\framebox(360, 85){
	\minibox{
	\textbf{Sprint 1}\\
	\begin{tabular}{p{3cm}p{8cm}}
	Scrum Master & Thiago\\
	Goal & Send and receive a message to the backend's server\\
	Sprint length & 2 weeks\\
	Met & Yes
	\end{tabular}
	}
}

Sprint 1 started off with setting up our working environment.
This included reading up on and configuring the tools we chose to use during the project,
which are described in \sect{tools}. The work of installing the tools was directed to the
ERNI group, which took over the tools for the beginning of the project. They had the
server in their room and they were installing all the tools for their project as well.

A major point for this sprint was also getting started with Scrum. It was very important
for our team to develop a culture, some sort of a Scrum routine that we were going to follow.
We felt we did a pretty good job doing this from the start of the project.

During the initial stages of development, We had a Skype meeting with Hugo Negrette and
Miguel Sosa, who worked on a similar project before. They wrote a Master Thesis\cite{masterthesis}
that was a fundamental starting point for our project, and they helped us understanding
critical parts of their code.

Finally, we implemented a simple application that could send a message to 
the Name Resolution Service (NRS) developed by the Backend team and receive an answer. As
we had more time than we expected, we started implementing parts of the Bluetooth communication and
some side functions for helping the other functionalities of our project.

\framebox(360, 100){
	\minibox{
	\textbf{Sprint 2}\\
	\begin{centering}
	\begin{tabular}{p{3cm}p{8cm}}
	Scrum Master & Thiago\\
	Goal & Communicate with another device using Bluetooth \& Publish and retrieve content with metadata\\
	Sprint length & 2 weeks\\
	Met & Yes
	\end{tabular}
	\end{centering}
	}
}

Now that we succeeded to communicate with the backend's NRS,
we needed to send messages according to the NetInf specifications.
Thus, we investigated the specifications and embedded those message regulations
into our code. 

We designed our first architecture draft based on OpenNetInf and implemented
the most important modules for sending/receiving messages to/from the NRS -
this time using OpenNetInf. We managed to share Information Objects between
Android phones using Bluetooth.

At this stage, the application could request a content using a short hash.
The specified hash was sent to the NRS and received a list of locators, that own
the requested object, in return. The locators were Bluetooth MAC adresses.
The application was then starting a Bluetooth discovery in order to find out
whether one of these locators was within reach. If so, a Bluetooth communication
to that locator was established. The hash was sent to the connected device, which
replied with the file that was identified by the hash.

The efforts at the end of the sprint were focused on the message that was being shared
and we spend two days creating the objects and making sure the under-the-hood part of
our application was working. During this time, we worked on creating the Metadata and a
parser for it, building the messages in a correct manner and checking where files were
being saved and retrieved from. We also worked on getting multiple types of files available
for visualization, leading us to work with the Gallery supported by the Android phones.
We ended up using the files' mime-type to open them.
\todo{cite netinf specs}

\framebox(360, 98){
	\minibox{
	\textbf{Sprint 3}\\
	\begin{tabular}{p{3cm}p{8cm}}
	Scrum Master & Paolo\\
	Goal & Successful presentation. Search content, cache and retrieve content from the NRS, implement a minimal web interface\\
	Sprint length & 3 weeks\\
	Met & Yes
	\end{tabular}
	}
}

Within the first week we mainly worked on the presentation for our review at Ericsson in Kista.
We prepared a paper prototype (consisting of screen shots of the final application) of our 
project idea of creating a browser application that we later called
Elephant.
The application should look like a normal browser and behave as expected. The main difference
we wanted to achieve is that the browser uses information-based networking instead of location-based
networking. Our idea was to create a browser using NetInf in a way, that the user does not
need to know what actually happens. It would be a first step of changing our way of 
networking. \textit{Note:} Using the paper prototype helped us a lot to agree on our
project in the end. Our client could understand our ideas and was very happy to
see where the project was going.

Before continuing adding new functionalities, we took some time during the third sprint
to clean up and refactor the code. We restructured our git branches due to our
experience we gained from the previous sprints. The structure is described in \sect{git}.

The new functionalities we added at last were publishing (register the 
device as a locator), full-put (uploading the content to the NRS) and
searching for contents using URLs. Furthermore we added a Local Resoltion Service (LRS)
besides to the NRS. The LRS looked up a content in a local database, that we
designed and implemented within this sprint. 

Now our application could search for content within a local database and in a remote
NRS by a URL. The response was a corresponding hash, that identified a
web page associated to that URL. In case the NRS owned the web page that
was searched for, we directly retrieved that web page instead of a hash.
Using the hash, we could get a web page from the LRS or a list of locators
from the NRS. In case of a list of locators, the device would start to
connect to other devices and download the content through Bluetooth.
As soon as we retrieved the web page, we could register ourself as a 
locator in the NRS or even upload the content to the NRS.
At this stage we displayed simple HTML web pages within a WebView environment, without any pictures or
java scripts.

It is worth to note that the search implemented in this stage was not developed in the NetInf
standards, not being connected to Resolution controllers. This was due to a perceived complexity of the code
and the short amount of time we had to finish the sprint. It worked as if it was integrated to NetInf, but 
just not glued to the internals of the service. We left this to be completed in the next sprint.

\framebox(360, 75){
	\minibox{
	\textbf{Sprint 4}\\
	\begin{tabular}{p{3cm}p{8cm}}
	Scrum Master & Kim-Anh\\
	Goal & Higher granularity browsing\\
	Sprint length & 2 weeks\\
	Met & Yes
	\end{tabular}
	}
}

As requested from Ericsson, we separated our application into two
applications: Elephant, our browser that uses the services of another application,
the NetInf services. 

In sprint 4 we created our final application. This included
creating the minimal browsing functionalities: Handling clicks on links
as well as displaying web pages correctly as they are displayed in
other browsers. The main challenge was to intercept the Android WebView
to redirect resource requests to our NetInf services, instead of simply
downloading them from uplink.

We also intregated the search functionality to NetInf, making use of the Search Controller and
implementing our search functionalities as Search Services.

Furthermore we needed to add settings and help pages. The user should
be able to decide which NRS to connect to and whether she wants to upload content to the NRS or
register her device as a locator.

After sprint 4 we had our final application that offered a browsing functionality
based on NetInf. Since some bugs were left to be reviewed, we tackled these
in the upcoming sprint.

\framebox(360, 85){
	\minibox{
	\textbf{Sprint 5}\\
	\begin{tabular}{p{3cm}p{8cm}}
	Scrum Master & Linus\\
	Goal & Running application without bug (not necessarily with Bluetooth Convergence Layer)\\
	Sprint length & 2 weeks\\
	Met & Yes
	\end{tabular}
	}
}

During this sprint we tried to fix a bug that prevented some pages to display correctly.
The bug showed up during the previous demo, so the goal of this sprint was to mainly
solve this bug and make sure we had a working application. After several tests we spotted the problem that was causing
the corrupted pages. The problem resided in the way we managed the search function and the messages sent to and from the NRS,
causing the loading of the wrong identifier of the seeked web page after a search. The backend could solve the problem easily and our application
started working as expected. During this iteration we also noticed some slow downs due to how we used the timeouts
in our network requests, so we fine tuned them to get a decent speed when browsing web pages.

Afterwards we discussed what to evaluate and how to evaluate our application. We then implemented logging of network
activities by storing each request with the relative type of transmission used (uplink, NRS cache, bluetooth or database).
In order to produce some statistics we also stored the amount of data that is transferred for each request and the time that
it takes to transfer the data. To simplify the test we added a functionality to automate the loading of several web pages
in sequence, so that we could gather a considerable amount of data from many web pages during each run of the application.

At this point we had a working web browser that used NetInf services and could support four different types of transmission link.
The next and last step would be to run the automated tests and analyze the obtained results, so that hopefully we would have interesting material
to show in the final report.

\framebox(360, 85){
	\minibox{
	\textbf{Sprint 6}\\
	\begin{tabular}{p{3cm}p{8cm}}
	Scrum Master & Harold\\
	Goal & Sucessful presentation and a product and course report\\
	Sprint length & 2 weeks\\
	Met & Yes
	\end{tabular}
	}
}

The last sprint involved the preparation of the final presentation and writing the final course report and
product report. As usual we devided the work into different tasks so that we could write different parts
of both reports simultaneously. Some things we did in the previous weeks were hard to remember, so we made
use of the notes we had in our logbooks and from the logs recorded in Jira. The difference between the course report
and the project report is the target reader. The intended audience of course report is teachers and future students
of this course, while for the product report the intended reader is the customer. The course report contains information
on how we got to the final product, while the product report focuses more on the product itself, the implementation,
how to run it and how to continue the development.

In this sprint we also run the automated tests we had implemented in the previous sprint and created tables
and graphs to use in the product reports. The final presentation was made with Prezi (reference?), an online tool
to produce professional presentation.

\subsection {ERNI}

\framebox(360, 85){
	\minibox{
	\textbf{Sprint 1}\\
	\begin{tabular}{p{3cm}p{8cm}}
	Scrum Master & Knut\\
	Goal & Send and receive a message to the frontend's client\\
	Sprint length & 2 weeks\\
	Met & Yes
	\end{tabular}
	}
}

During the first two-week sprint the ERNI team mainly setup their environment along with coordinating with LISA on building a server which would communicate to a client using simple (non NetInf messages). The code from this demo became the reference for the real NetInf server which was developed during the second sprint.

\framebox(360, 102){
	\minibox{
	\textbf{Sprint 2}\\
	\begin{centering}
	\begin{tabular}{p{3cm}p{8cm}}
	Scrum Master & Knut\\
	Goal & Implement NetInf Publish and Get content with metadata\\
	Sprint length & 2 weeks\\
	Met & Yes
	\end{tabular}
	\end{centering}
	}
}

In the second sprint which ran for another two weeks, ERNI created the basic NetInf server which was able to use the appropriate Publish and Get messages from the NetInf protocol draft. A major part of the sprint was devoted to creating the Message handler and the formatter which would accept messages passed in from the HTTP handler and convert them to the internal representation of a NetInf message. The ERNI team also developed a very basic list database in order to store information. 


\framebox(360, 108){
	\minibox{
	\textbf{Sprint 3}\\
	\begin{tabular}{p{3cm}p{8cm}}
	Scrum Master & Faroogh\\
	Goal &  Implement NetInf Search ,binary caching and retrieve content, implement a real database(riak) and implement a plug and play database wrapper\\
	Sprint length & 3 weeks\\
	Met & Yes
	\end{tabular}
	}
}

The third sprint consisted of refining the Publish and Get messages, implementing search and binary storage and finally introducing a real database. The client also stressed that they would like the ability to easily attach other databases, thus a database interface was created and the ERNI team attached Riak to the NetInf system.

\framebox(360, 85){
	\minibox{
	\textbf{Sprint 4}\\
	\begin{tabular}{p{3cm}p{8cm}}
	Scrum Master & Jon\\
	Goal & Provide the client with a draft of the NetInf Video Streaming\\
	Sprint length & 1 week\\
	Met & No
	\end{tabular}
	}
}

During the fourth sprint the ERNI team had choose to diverge from the LISA team since they had finished all the functionality described in the draft document and no longer required to build features for LISA. The client was impressed and suggested that the team look into applications for the NetInf system. The major application that was discussed was the use of NetInf and Video streaming to alleviate congestion problems while broadcasting content. The goal here was to provide the client with a draft of the new protocol but there were many disagreements on what the draft should contain and the team missed the deadline to hand in the draft.

In an effort to fine tune our sprint planning, the team decided to try one week sprints in order to see how much we can accomplish with a smaller amount of time. This was due to the two-week sprints feeling too short leaving the team to scramble

\framebox(360, 98){
	\minibox{
	\textbf{Sprint 5}\\
	\begin{tabular}{p{3cm}p{8cm}}
	Scrum Master & Jon\\
	Goal & Implement NetInf video streaming on top of the existing NetInf NRS code and UDP convergence layer\\
	Sprint length & 1 week\\
	Met & Yes
	\end{tabular}
	}
	
}

In the fifth sprint the ERNI team had gotten the approval of the client to start implementing the NetInf video streaming protocol. It was a one week sprint where the team also added the UDP convergence layer after meeting with the client in the last sprint.


\framebox(360, 80){
	\minibox{
	\textbf{Sprint 6}\\
	\begin{tabular}{p{3cm}p{8cm}}
	Scrum Master & Marcus\\
	Goal & Fixing bugs within the system, implementing feedback from client on video streaming\\
	Sprint length & 2 weeks\\
	Met & Yes
	\end{tabular}
	}
}

In the sixth sprint the ERNI team mostly fixed the major bugs in the system and finalizing the NetInf video streaming, the major feedback here was 

The final sprint(sprint 7) both teams joined together and had one large sprint writing the course and product reports.
