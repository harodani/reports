\lstset{language=Java}

The purpose of this section is to present the code convention used in the implementation of the Android-based application. These code standards are based
on the Android Code Style Guidelines for Contributors\footnote{\url{http://source.android.com/source/code-style.html}}.

\section{Java Language Rules}

\begin{itemize}
  \item Do not ignore exceptions. It is only acceptable to ignore exceptions if there is a good reason to do it, which should be given as a comment in the code. As a general rule, always:
  \begin{itemize}
   \item handle the exception;
   \item or throw a new exception according to the level of abstraction;
   \item or handle it gracefully;
   \item or throw a new \textit{RuntimeException} in case there is nothing possible to do.
  \end{itemize}
  \item Don't catch a generic exception e.g. \textit{Exception e}. Alternatives to do this are:
 \begin{itemize}
    \item Catch each exception with a separate catch block after a single try.
    \item Refactor the code in order to have multiple try blocks.
    \item Rethrow the exception and let it be handled on the next level.
  \end{itemize}
  \item Do not use finalizers. Let the garbage collector do its job.
  \item Always write full imports.
  
  \section{Java Style Rules}
  \item Use Javadoc standards for commenting code. Always write the descriptions in third person. This rule can be skipped if the method is too trivial e.g. a \textit{getters} and \textit{setters}.
  \item Try not to exceed 40 (forty) lines of code for each method.
  \item Try to keep the scope of a variable as small as possible. Also, try to initialize it with a proper value. 
  \item Order the imports beginning with Android libraries, followed by third parties libraries and ending with \textit{java} and \textit{javax} classes, each group separated by an empty line.
  \item For indentation, use four spaces for blocks and eight spaces for line wraps (i.e. when the line of code is too long and needs to be cut).
  \item Limit the line length to 80 (eighty) characters. 
  \item Use standard Java Annotations e.g. \textit{@Deprecated, @Override, @SuppressWarnings}.
  \item Use acronyms as words e.g. write \textit{XmlHttpRequest} not \textit{XMLHTTPRequest}.
  \item Use TODO comments for temporary fixes and future work.
  \item Use all five levels of logging (Error, Warning, Informative, Debug and Verbose) as applicable.
  \item Use the standard brace style: 
  \begin{lstlisting}
      public void foo() {
	  if (...) {
	      doSomething();
	  }
      }
  \end{lstlisting}
  \item Use the following naming conventions:
  \begin{itemize}
    \item Non-public, non-static field names start with m.
    \item Static field names start with s.
    \item Other fields start with a lower case letter.
    \item Public static final fields (constants) are letters in upper case and spaces replaced by underscores.
    Example:
      \begin{lstlisting}
      public class MyClass {
	  public static final int SOME_CONSTANT = 42;
	  public int publicField;
	  private static MyClass sSingleton;
	  int mPackagePrivate;
	  private int mPrivate;
	  protected int mProtected;
      }
      \end{lstlisting}
    \end{itemize}
  \end{itemize}
