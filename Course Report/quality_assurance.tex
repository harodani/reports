\section {Quality Assurance}

In large projects you need to be able to control the quality of your product. Does it do what the customer asked? Is it bug free? Is it releasable? while some companies have specific quality assurance teams, this instance of the course usually does not and the team had to explore ways of answering the above questions.

While SCRUM and Agile methodology easily answers the Is it releasable question, just because you implement SCRUM and Agile does not mean you are doing it correctly. In SCRUM, at the end of every sprint you should have something that is demo-able. Sometimes this is not the case(failed sprints, outstanding issues etc..)

\subsection{Pair-programming}

The ERNI team decided to take the pair programming approach to quality assurance. Two programmers complete a task and two other programmers review that work for bugs, imperfections(according to code standards), and integration status. 

The team felt that this worked well as we were eight programmers who had little to no experience in functional programming and large group work projects. Often bugs were caught way before the task was completed and put into the review process. 

Using pair programming also helped develop our skills faster as we taught each other how the language worked and had knowledge redundancy ( if one person was away the task would not be stopped). After review you had twice as many members of the team having full knowledge of the code and functionality introduced into the product.

\subsection{Tool-code review}

The LISA team decided to use a tool called Gerrit to preform code reviews. However this proved to be difficult and time consuming to setup and in the end only introducing many problems in the workflow. 

While a tool for code review is supposed to be powerful in a teams repertoire, ineffective use of it can lead 
large overhead that can delay the progress of a project. If a tool can not be understood and setup in about a day we recommend dropping it and search for another (if required). Often times the simplest solution is the most effective, in this case the LISA team eventually switched to using Pair-programming exactly like the ERNI team.


