\section{Use of SCRUM in this project}
Throughout the project we tried to apply the Scrum methodologies thoroughly.
We arranged the room and the desks to facilitate direct communication among team mates,
and we set up a whiteboard for the post-its. 
In the frontend team we agreed on that each team member should try to
be the Scrum Master at least once during the whole duration of the project.
We could take advantage of the fact that the Project CS is only a course within university
so that we didn't need to have only one Scrum Master.
Since we were two teams, we also always had two Scrum Masters, one for each team.

\subsection{Daily meetings}
Due to the separation of the group into two teams, we had two distinct sets of meetings in the two offices.
Daily meetings for the frontend team took place everyday at 9:00 o'clock sharp and time boxed to 15 minutes,
while at 9:15 the standup meeting would start for the backend group.
During both meetings the other team's Scrum Master would attend the other team meeting and vice versa.
The reason for this was to facilitate synchronization between both teams.

In our opinion these meetings were of great value. 
They encouraged everyone to actively talk to the other team members about
the status and their problems.
Each stand up meeting was mandatory so that we could be sure
that everyone would start working at the time. 
Moreover it was a good way for planning short tasks and catching bad decisions early.

\subsection{Sprint planning}
After each demo we planned the sprint for the next iteration, trying to get done before the weekend.
In this way we started working directly in the beginning of the new week. Most of the times we had demos on Thursdays,
so that we could do a pre-planning on the same day and meet the customer on Friday to get the planning approved.
For the estimation of the available working days for each sprint, we calculated the total amount of days
according to the number of the team members, and used a productivity factor of 0.7. A team of 5 developers working
for two weeks would have 5*10*0.7 available story points. One story point corresponds to one man day (8 hours). 
In order to estimate the workload of each story we used planning poker, where each individual has a set of cards
with numbers and votes with a card giving points to each story. 

In our case, the requirements were not set from the beginning. Therefore we had a continuously
changing product backlog, which was updated by us at each pre-planning phase.

\subsection{Demo}
Demos were held at the end of each sprint during which we showed our results.
We demonstrated the application to Olle, Muneeb and Ericsson in a very relaxed environment. Having short sprints and doing frequent
demos ensured that Ericsson saw our progress often.

\subsection{Retrospectives}
Retrospectives were done in form of meetings after each sprint. Each member had to say three good things and three bad things about
the completed sprint. It was a great way to point out what was positive during the sprint
and to expose constructive critics for the bad parts. In this way we could always try to improve our process in the next sprint.

\subsection{Conflicts}
Stupid kids screaming all the time.
