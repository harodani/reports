\section{Use of SCRUM in this project}
Throughout the project we tried to apply the Scrum methodologies thoroughly.
We arranged the room and the desks to facilitate direct communication among team mates,
and we set up a whiteboard for the post-its. 
In the frontend team we agreed on that each team member should try to
be the Scrum Master at least once during the whole duration of the project.
We could take advantage of the fact that the Project CS is only a course within university
so that we didn't need to have only one Scrum Master.
Since we were two teams, we also always had two Scrum Masters, one for each team.

\subsection{Daily meetings}
Due to the separation of the group into two teams, we had two distinct sets of meetings in the two offices.
Daily meetings for the frontend team took place everyday at 9:00 o'clock sharp and time boxed to 15 minutes,
while at 9:15 the standup meeting would start for the backend group.
During both meetings the other team's Scrum Master would attend the other team meeting and vice versa.
The reason for this was to facilitate synchronization between both teams.

In our opinion these meetings were of great value. 
They encouraged everyone to actively talk to the other team members about
the status and their problems.
Each stand up meeting was mandatory so that we could be sure
that everyone would start working at the time. 
Moreover it was a good way for planning short tasks and catching bad decisions early.

\subsection{Sprint planning}
After each demo we planned the sprint for the next iteration, trying to get done before the weekend.
In this way we started working directly in the beginning of the new week. Most of the times we had demos on Thursdays,
so that we could do a pre-planning on the same day and meet the customer on Friday to get the planning approved.
For the estimation of the available working days for each sprint, we calculated the total amount of days
according to the number of the team members, and used a productivity factor of 0.7. A team of 5 developers working
for two weeks would have 5*10*0.7 available story points. One story point corresponds to one man day (8 hours). 
In order to estimate the workload of each story we used planning poker, where each individual has a set of cards
with numbers and votes with a card giving points to each story. 

In our case, the requirements were not set from the beginning. Therefore we had a continuously
changing product backlog, which was updated by us at each pre-planning phase.

\subsection{Demo}
Demos were held at the end of each sprint during which we showed our results.
We demonstrated the application to Olle, Muneeb and Ericsson in a very relaxed environment. Having short sprints and doing frequent
demos ensured that Ericsson saw our progress often.

\subsection{Retrospectives}
Retrospectives were done in form of meetings after each sprint. Each member had to say three good things and three bad things about
the completed sprint. It was a great way to point out what was positive during the sprint
and to expose constructive critics for the bad parts. In this way we could always try to improve our process in the next sprint.

\subsection{Conflicts}
In any large group project there will be many different personalities, opinions, experiences and backgrounds. The majority of people do not have very much experience in school to work with large groups(10+) people. This course forces the members to work together and to do it in a specific project management style(Scrum). It is inevitable that there will be conflicts and disagreements. 

The following are some points that the teams followed when we had such conflicts.
\begin{itemize}
\item Try to address the problem with the individuals involved together.\\ \\
The first and foremost approach is to try and talk it out with just the people involved. Please note that you may have very different personal and work experiences and while those are great to come with sometimes things are not set in stone, what you have done in previous course and workplace may not apply very well to the particular project. It is best to come to the project with an open mind and offer your past experiences as a way to suggest how to go about preforming certain tasks, but never blatantly ignore everyone else because you feel they are wrong. This course is a team-effort based project and not to be treated as a workplace, here you are allowed and even encouraged to make mistakes and to learn from them.

\item Realize that just because you do a task does not mean that it was the right one.\\ \\
The idea of Scrum, simply put is change the product during the development cycle quickly. As developers we may sometimes get attached to the code we have written and can take offence when the team or client decides to remove the code. In this case it is not a matter of the team or client attacking you, it is a matter of whether or not the code or task is still relevant to the project after a design change. The project will be very different from the beginning to the end. Code will be obsolete and deprecated at certain points, just live with it and continue contributing and understand that your work is still recognized by everyone else.
 \\
\item Use the Scrum master as a mediator.\\ \\
In the case where addressing the problem yourself did not work out, the use of Scrum in this project gives everyone a great resource, the Scrum Master. It is the Scrum Masters duty to remove impediments in the sprint, however big or small. Conflicts are a great impediment and can sometimes lead to failed sprints and worse de-motivation of the team. The Scrum Master should in this case act as an impartial mediator and come up with a solution to the problem, even if it means making sure the conflicting personalities do not work together in the end if it can be avoided.

\end{itemize}
