\section{Use of SCRUM in this project}
Since the project start we tried to apply the gained knowledge we got from studying Scrum.
We arranged the room and the desks to facilitate direct communication among team mates,
and we set up a whiteboard for the post-its. We decided also that each team member should try to
be the Scrum Master at least once during the whole duration of the project.
Since we were two teams, we also always had two Scrum Masters, one for each team.

\subsection{Daily meetings}
Due to the separation of the group into two teams, we had two distinct sets of meetings in the two offices.
Daily meetings for the frontend team took place everyday at 9:00 o'clock sharp and time boxed to 15 minutes,
while at 9:15 the standup meeting would start for the backend group.
During both meetings the other team's Scrum Master would attend the other team meeting, and vice versa.
The reason for this was to facilitate synchronization between the two teams.
We think that these meetings was of great value as they could encourage everyone to say something and
expose their problems during the project. Also it was a good way for planning short tasks and catching bad decisions
early.

\subsection{Sprint planning}
After each demo we did the sprint planning for the next sprint, trying to get done before the weekend.
In this way we could start working directly in the beginning of the new week. Most of the times we had demos on Thursdays,
so we could do a pre-planning on the same day and meet the customer on Friday to get the planning approved.
For the estimation of the available working days for each sprint, we calculated the total amount of days
according to the number of the team members, and used a productivity factor of 0.7. A team of 5 developers working
for two weeks would have 5*10*0.7 available story points. To estimate the workload of each story we used poker cards,
where each individual would vote with a card giving points to each story.

\subsection{Demo}
Demo were held at the end of each sprint during which we showed the result of the work in the form of a working application.
We showed the application to the teacher and the customer in a very friendly environment. Having short sprints and doing frequent
demos would ensure that the customer could see progress often and the chances that he/she gets happy are high.

\subsection{Retrospectives}
Retrospective were done in form of meetings after each sprint. Each member had to say three good things and three bad things about
the completed sprint. It was a great way to point out what was positive during the sprint
and to expose constructive critics for the bad parts. In this way we could always try to improve things in the next sprint.

\subsection{Conflicts}
Stupid kids screaming all the time.
