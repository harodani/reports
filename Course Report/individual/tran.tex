\section{Kim-Anh Tran}
In the first sprint I read up on JIRA and presented the ways to use
JIRA within our project. Afterwards I mainly worked together with Paolo on establishing
the first Bluetooth communication, so that we could handle file requests from other connected
devices. %%that were answered by file transfers.

During the second sprint Harold, Paolo, Linus and I created the first draft of our architecture using OpenNetInf. Thereafter Paolo and I continued on our previous Bluetooth implementation. We added
the Bluetooth discovery and more importantly the functionality to request and
transfer NetInf NDO between devices.

The third sprint I worked on reconfiguring Jenkins with Git and writing a document on
our git workflow. I added a test project for our code that was run by Jenkins.
Afterwards Harold and I worked together in order to develop a database for storing
NDO and thus a Local Resolution Service.

Sprint four involved separating our current application: one that provides
only the NetInf services and one that contains our application which uses
these services. With help from Paolo I separated the two projects and resolved
all dependencies. Afterwards I joined Paolo in order to finish parsing HTML files
to be properly displayed within our WebView component while using our NetInf services.
%%More specifically, we intercepted the WebView activity in case it attempted to
%%download a resource of a page (such as an image) and instead route the 
%%request to our NetInf services.
During this sprint I was Scrum Master. Amongst other tasks I needed to attend the other group's
stand-up meeting, create the stories and tasks in JIRA and update our Scrum board.
It was a good experience to take the role as a Scrum Master and to organize our Sprint.

Within the last sprint I solved the text encoding problem when displaying 
a number of web pages. I cleaned up our logs so that they were readable and
more useful during debugging. Finally I created a JAR file for
libraries that were used in both applications.

