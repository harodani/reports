\section{Knut Lorenzen}

In the pre-SCRUM phase I made two project proposals: One was to integrate NetInf into the Android OS, the other one to start a new NetInf branch in Erlang. As Linux kernel level development was considered tedious the first idea was dropped immediately. The Erlang idea initially received a luke warm response as it did not sound very original compared to other proposals. However our Ericsson contacts mentioned that they would love to see a NetInf implementation in Erlang as it is ``their`` very own language, and after a few days more and more group members changed their mind, perhaps because they realised it would be a great opportunity to practise and learn Erlang on a real project.

I became the SCRUM master of the backend group for the first two sprints since nobody else volunteered. Having worked as a software developer for a few years after my graduate studies and therefore being more experienced, I felt that I would be more useful in overseeing the development process rather than writing code. During that period I spent my time curating the backlog, reviewing people's tasks, coaching them in working test-driven and setting up and using the development infrastructure. I did not write a single line of code until the third sprint. I also presented the backend group's work during first review at Ericsson after sprint 2.

Unfortunately for me (I had really enjoyed being the SCRUM master), Olle, the course teacher, requested both teams to appoint new SCRUM masters after that, so that others could have the opportunity to practice that role. In the three weeks of sprint 3, my work focussed around creating an integration test. So far, only module tests existed, and the HTTP interface was tested interactively using curl. I tried hard to convince people to write tests for the (automated) integration test rather than interactive curl-tests or module tests. I continued to add test cases to the integration test and improve it until the end of the project

In sprint four, I decided to become an additional spokesperson for the backend team. I had the impression that there was a lack of communication between the product owners and us during the previous sprint, i.e. none at all until the demo. This had led to some members implementing features not requested by the product owner. At that point, we had mostly finished implementing the NetInf protocol draft, and the focus of the project changed towards adding streaming functionality. For this, no specification or prior implementation existed. I tried to develop a design draft together with B\"orje, our contact a Ericsson, through email communication. This did not work out very well, and so we decided to come up with our own design and present it at the demo after sprint five.