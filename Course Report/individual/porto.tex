\section{Thiago Costa Porto}
When the project started, I read a few papers about ICN and tried to focus on the issues that we were going to face going forward. We had meetings to decide what our project would be like and I felt I was very active in those. One thing that helped was my previous experience with Scrum, which lead to greater understanding of it by doing this in the university level, giving you the chance to try things the proper way, so to say. When time came, I chose the frontend team because I thought it would be fun to work on the ``client'', shaping it to the way that we had planned.

I was the Scrum master for the first two sprints. In the first sprint, I set up Jira, our project tracker tool, and did all the things the Scrum master should do. I was very focused on getting Scrum to work, at the same time I coded small features for our client. I focused on the networking side of the client, and spent several hours understanding how Android works and how the code we had at our disposal worked. In the second sprint, I could focus more on developing the application and less on Scrum -- everything was going as it was supposed to go -- and I wrote a few of the classes used on the client's backend. I also provided support for measurements (download/upload) and added early support for metadata during transfers.

On the third and fourth sprints I focused mainly on getting the search functionality to work properly. In the third sprint, I provided a solution that was out of the 'NetInf' architecture, and I focused on integrating it to NetInf using the Resolution Services on the fourth sprint, together with Linus. Apart from the search, I also refactored some code and helped document our code, helped a little with the design of the database, started user feedback and setting up our revised workflow. In the fourth sprint, I worked a lot with JSON, communication with the NRS and making responses uniform throughout our application, with Linus.

Near Christmas, I defined the Evaluation with the other team members and started working on that. I implemented the logging functionality together with Linus and did some extra refactoring on the side. Close to the finish line, I spent some more time documenting the code for further usage.

This was a very interesting course because it not only simulates a work environment, but provides a lot of insights on your own behavior and on team management. It is very good working with people from different places and with different backgrounds. I definitely recommend this course.

