\section{Faroogh Hassan}

During the first sprint the whole back-end group concentrated on learning Erlang and Open Telecom Platform (OTP). A two day workshop on Erlang by Henry Nyström of Campanja and another workshop by Gustaf Naeser (Hansoft) on Agile software developmen were also part of this sprint. I worked with Jon to establish the coding standards which we followed throughout the rest of the project. Me and Jon also investigated if their are any Extreme Programming (XP) practices that we can incorporate in our development methodology and we came up with a set of recommendations. \\

I started the second sprint by taking part in developing the overall design of our application. I also worked with Thomas to write code and some tests for message handler and message parser.\\

Third sprint was the longest sprint of this project (3 weeks) and I got the opportunity to be the Scrum master of the back-end team for this sprint. As a Scrum master my major task was to remove any impediments that may arise during the course of development. I worked with Alex on writing for the search functionality. I also worked with Kiril on the NRS discovery protocol where we used UDP messages for the discovery purpose. Apart from that I also wrote unit tests for the modules where unit tests were missing. \\

In the fourth sprint I was involved in designing the architecture of the streaming video functionality. We wrote a draft specification for this functionality based on the proposed architecture. \\

In the fifth sprint I investigated if our implementation of Netinf protocol is compatible with other implementations. We aborted this task later on customer's request. \\

Sixth sprint was the last sprint before the Christmas break where we made some final adjustments to our application and cleaned up our code. I wrote 'specs' and comments in the modules where they were missing. \\

Seventh sprint was the last sprint of our project and we dedicated the whole sprint to write course and product reports. \\