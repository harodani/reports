\section{Linus Sunde}

During the first sprint I attended the Erlang Workshop which was mainly directed towards the backend group. This made it natural for me to work with integration between the groups. Together with Thomas and Marcus I worked on sending NetInf messages using HTTP between our application and their server.

In the second sprint I sat down with Jon and Alex and discussed the protocol draft for the NetInf HTTP convergence layer. This discussion resulted in an initial specification. The frontend group decided to use OpenNetInf in our application. I spent some time incorporating OpenNetInf into our project. After this I worked with Harold to create the Resolution Service which communicates with the backend's NRS. This also involved working together with Thomas and Marcus from the other group to solve integration issues as they popped up. Integration issues kept popping up during all the coming sprints and I spent a huge amount of time working on these, most often together with someone from the other group.

I started sprint three with creating a paper prototype together with Kim and Paolo, in preparation for the first review in Kista. The review went well and I felt the paper prototype really gave us a more concrete feeling of our goal. After the review I looked into speeding up the compile time of our application as it was making testing of small changes slow and painful. I also spent a lot of time refactoring, which in hindsight was time well spent. As for new functionality, I worked together with Paolo creating some setup dialogs and downloading of simple web pages using NetInf, as well as sending and receiving cached files from the backend's NRS. Finally I read up on testing using Android JUnit and created tests for some of my code.

During sprint four I once again spent time refactoring, and once again I feel it was time well spent. I worked with Harold to create a settings menu for our application and to make the program use these settings. I worked with Paolo to create a way for our application to know if files were acquired using the Internet, Bluetooth, the NRS cache, or the database on the phone.

I was Scrum Master during sprint five. I spent some time making sure some backend fixes we needed were implemented by the backend. Other than that I mainly worked on preparing evaluation together with Thiago and Paolo. We implemented logging functionality to be able to gather some statistics for our report.

The last sprint was dedicated to writing the reports.
