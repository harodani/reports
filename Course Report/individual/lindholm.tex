\section{Alexander Lindholm}
The first few weeks of the project were all about team-building as well as reading up on the subjects ICN and NetInf. The first sprint was mainly about refreshing our functional programming skills as well as setting up all the tools such as Git, Redmine, Jenkins, Emacs etc. We also came to certain agreements such as trying to use test-driven-development. 

The second sprint I was involved in the creation of a external protocol draft as well as setting up Jenkins along with Kiril and creating Curl scripts for testing of our system along with Jon. Jon and I also created a first naive version of a storage. At the end of sprint most of us were busy fixing bugs before the presentation at Ericsson in Kista. Here I also spent time on implementing a "beautiful logging system" for the presentation in Kista.

Sprint three started off with the presentation in Kista and it went well. Everyone was pleased with how the presentation went. During the rest of the sprint I mostly worked with Kiril and we were involved in writing a module for validation of hashed content as well as writing code for merging of NDO's in the database as well as fixing the tests for the storage module. Kiril and I also created the plug-n-play database architecture. Other than that Faroogh and I implemented searching of NDO's within our system.

Sprint four I developed modules for forwarding of messages on the HTTP convergence layer.
A few days after the sprint had started the backend team started to diverge from the frontend due to several facts;
we were done with the basic NRS functionality in the backend and we were planning on implementing streaming within NetInf, which was something that the frontend didn't have time to do. 
Therefore I mainly worked with Danielle and Jon this sprint and implemented an HTTP-client that would come to be used mainly for streaming, but also worked as a fully functional NetInf-client. 

During sprint five I mainly worked with bug fixing within the system as well as making the HTTP-client more robust and fully functional.

During sprint six we had feature-freezed and my work mainly consisted of fixing bugs as well as refactoring code. A lot of time was spent on fixing a bug that resolved around multipart-Http-data being corrupt after transmission, it turned out that the bug was within Cowboy, the open-source Http-client we used. Kiril and I also created the initial structure for the reports. 

In the beginning of sprint seven I worked with Jon on the implementation of a different version of NetInf-streaming that uses pure NetInf for all transmissions instead of a mix of NetInf-messages along with a content-handler which were used in the previous streaming version. 

