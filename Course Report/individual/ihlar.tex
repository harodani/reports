\section{Marcus Ihlar}
During the first I studied OTP design principles, investigated existing http libraries for Erlang and implemented a simple demo server together with Thomas. We decided to use Cowboy for http handling. The server partially implemented NetInf publish and get functionality.

The second sprint signaled the start of real product development. In the beginning of the sprint I did an overall design of the intended system with Faroogh and Thomas. When actual coding started I wrote a lot of the boilerplate code necessary to setup an OTP application, later I focused on event-handling logic and test code.

In sprint 3 I implemented content storage together with Kiril, after that we focused on getting the integration test working properly. Designed the architecture for forwarding of NetInf messages together with Thomas and Alex. I implemented message id storage (as part of the distribution architecture) together with Thomas and then updated the event handler to handle message forwarding, this led to alot of re-factoring throughout the system, especially to make message passing asynchronous. 

Sprint 4 was short, only one week. I focused on re-factoring and code cleanup, especially in the http message formatting module. We also did some backlog grooming and at the end of the sprint I helped finish the streaming draft.

During the fifth sprint I implemented streaming functionality together with Jon in accordance to the draft written the week before. 

Sprint 6 was my turn to try the role of scrum master. Being scrum master at this point felt very straight forward since we were so far into the project and had a good working environment. I did a lot of bug fixing and work on the integration test. Me and Thomas rewrote some of the streaming code and started implementing pure NetInf streaming. We ended the sprint with a beer and Quake 3 session!

I started sprint 7 by finishing the pure NetInf streaming in order to be able to run evaluation tests. I also worked on presentation, evaluation and reports.
 

 



