\section{Daniele Bacarella}

Sprint 1:
I setup my working environment by installing all the software needed to start developing such as the Erlang compiler along with the build tool Rebar and the editor Emacs.
Once the environment was ready I started studying and practicing Erlang while getting familiar with the editor Emacs.

Sprint 2:
Kiril and I both worked on the NRS logging and the HTTP handler for the system. Then we researched GIT practices and came up with our own workflow and taught it to the group.
Afterwards I created the inital version of the project Makefile and integrated Rebar into it.

Sprint 3:
Jon and I researched database alternatives to the one already adopted that uses an Erlang internal data structure which is a list of key-value pairs. For obvious reasons it did not represent a valid solution to store data since it did not provide reliability and persistency along with other concurrent features that regular DBMS provide.
Among the many available options, we chose Riak.
After having set everything up to make it fully working we proceeded writing some tests and the database wrapper for it.
Finally we wrote a install guide on the wiki. 

Sprint 4:
The back-end team started working independently from the front-end team on the new NetInf NRS Streaming feature.
It required us to implement an HTTP client interface to the upcoming NetInf Nrs Streaming.
During this sprint I worked with Alex to create the first prototype of the web interface and the Http client which would communicate with the running NRS to perform the operation requests issued by the user.
We also researched video players suitable for the web page.

Sprint 5:
I worked with Alex and Jon fixing bugs and design in the Http client and the web interface.

Sprint 6:
I helped Kiril with starting the Product and Course reports as well as finalizing the HTTP client interface with Alex.

Sprint 7:
I started writing the Product and Course reports.
