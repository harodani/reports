\section{Harold Martínez}

In the first sprint, I worked defining the code conventions that we wanted to use. 
We discussed them, refined them and I set up the Eclipse plug-in Checkstyle with the chosen conventions.
Also, I was selected as the groups' spokesperson for the whole project, so I managed the communication 
between the customer and the group.\\

In the following sprint, Kim-Anh, Paolo, Linus and I worked on the first architecture draft. On the other hand,
I worked with Linus on the development of the earliest version of the Resolution Service, which connects with the 
backend's Name Resolution Service, implementing the \emph{GET} and \emph{PUBLISH} methods.\\

In the third sprint, I prepared, together with Knut and Thomas, the presentation for Ericsson. I also presented this. 
Then I worked together with Kim-Anh designing the Information Object database that will be used for the 
Local Resolution Service.\\

In the fourth sprint, Linus and I created the settings GUI for the Android application, letting the user change 
some configurations values. I also set up a GitHub account to share the evolution of our project. In order to improve 
the user experience in the application, I separated the Bluetooth discovery process and create a scheduled discovery that 
will be triggered from the moment the application starts, running every five minutes.\\

In the fifth sprint, I began defining and implementing the Bluetooth Convergence Layer, however, due to time constraints 
and other priorities, this job was not finnished. \\

I was SCRUM Master for the whole group for the last sprint where we wrote these reports. I was also in charge of presenting 
our results at the end of the course.\\

