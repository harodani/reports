\section{Kiril Goguev}

During the first sprint I helped setup the build tool environment (investigating Buildbot with Alex, Faroogh and Jon). Then, I read OTP, attended the Erlang workshop and started to get a grasp of Erlang.\\

In the second sprint, I setup Jenkins with Alex and connected the GIT repositories, I also created a document on how to set it up from scratch. 
Daniele and I designed and implemented the HTTP Handler and the NRS logging service. We also created the GIT practices document and taught the groups how to use the proper workflow for our project. Finally,
I along with Thomas helped design and implement the foundation of the NetInf protocol (The internal representation of messages in the system).\\


In the third sprint
Faroogh and I migrated all the data and build tools from the old server to a new server(due to errors).
Alex and I fixed merging the metadata in the list database, creating and implementing the plug and play (PNP) database architecture, abstracting the list database to use the new PNP database architecture and finally the content validation. 
Later, I worked with Marcus to implement content storage and content handler into our system as well as fixing the integration test which was broken when we added all the new modules. 
Finally towards the end of the sprint, Faroogh and I designed and implemented a UDP discovery protocol to be able to find other NRS' on the network.
At the very end of the sprint I started looking into Python SAIL implementation of NetInf but had to abandon it since there was too many problems to fix in order to be able to communicate(this was a problem of the differences in the drafts each implementation followed). 
I also wrote a wiki article for the plug and play database architecture.\\

This is the sprint where I felt that I had a good grasp of Erlang and how to code in the proper OTP way. \\

In sprint 4,
I had a part in designing the very first NetInf video streaming draft along with Faroogh and Knut. 
Later Faroogh and I added truncation to the NRS system and verified that we met all the required components of the netinf protocol draft provided at the beginning of the course by the client. \\

In sprint 5,
I designed a setup/install script.
I also designed and implemented the UDP convergence layer from the draft with Thomas and deprecated the UDP discovery protocol Faroogh and I coded in sprint 3. 
I also took over Faroogh's old task of making the list database persistent. 
Finally, I wrote the wiki article on how to use the script and test the UDP convergence layer. \\

In sprint 6,
Alex and I organized the Course and Product Reports into separate files.
Thomas and I made the system configurable using external files (.config files) that can be loaded on the command line at runtime and
I wrote the wiki article for how to write for the reports using the structure we created as well as how to use the configuration files in the system.
Also populated draft sections of the course and product report using all the wiki articles the backend team had written during the course with Daniele.
I fixed and finalized the setup/install script.\\

Finally in sprint 7,
I reorganized the structure for the reports into folders and showed people how to use the structure.
I wrote parts of the product and course report.\\
