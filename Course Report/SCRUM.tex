\section{Scrum}
The project generally specifies scrum project methodology. Scrum is a framework used when a team of people is involved in the development of complex projects. Scrum consists of self-organizing teams with associated roles. A team has the skills to accompish the work and scrum is designed to optimize flexibility and productivity. This methodology is rapidly becoming the standard in companies who develop software and makes sense for students who are about to go into the real world of software development to learn it. 

\subsection{Roles}

SCRUM distinguishes between the scrum master, the product owner and the development team.

Scrum master position means an individual who will be responsible for the product in a
particular sprint, as well as liaise between the team and the product owner.
A scrum master is responsible for making sure that the team adhere to the scrum guidelines,
and should never be seen as the boss of the group, but instead as the coach.
She should also make sure that the team is one right track keeping a constant eye at the goal
of the sprint and the definition of 'done' for the different tasks.

Product Owner position means an individual who is considered to be the client in the project.
The person who owns and controls the development of the software with the help of the backlog,
prioritizing features and generally the one who is supposed to give you the direction in sprint planning and demos.

The development team is the set of individuals who are working on the code for the product owner.
The team is responsible for creating increments and releasing a working version of the software
for each sprint.

\subsection{Scrum Keywords}
\begin{itemize}
\item{\textbf{User story}}\\
A user story is defined as a function the system must provide to the end user.
A user requirement is translated to a user story so that the developer does know exactly
what the user needs to accomplish for that function and why she needs it. User stories are usually
broken down to tasks and each task is the smallest work unit to implement.
It is important that the developer team agrees on the definition of "done" for each story.

\item{\textbf{Product backlog}}\\

\item{\textbf{Standup meetings}}\\
\item{\textbf{Sprint}}\\
\item{\textbf{Sprint backlog}}\\
\item{\textbf{Sprint planning}}\\
\item{\textbf{Srint demo}}\\
\item{\textbf{Sprint retrospective}}\\

\end{itemize}
\subsection{Daily meetings and Sprint planning}

Since we have two teams (ERNI and LISA) we had two distinct sets of meetings and sprints. This meant that we had to bear in mind what the other team was doing since we were supposed to work in sync. One scrum master for each team with constant communication between both teams. 

\subsection{Use of SCRUM in this project}

\subsection{Conflicts}
