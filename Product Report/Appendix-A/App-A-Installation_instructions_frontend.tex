This section will describe how to configure the environment to run the frontend application on an Android device. 
This description includes configuring Eclipse with Android in order to continue development. 
The guide assumes that the Java SDK is installed.

The frontend team have been working with:
\begin{itemize}
\item Eclipse Indigo, Service Release 2
\item Android Version 4.1, API Level 16
\end{itemize}

\subsection{Configuring Eclipse with Android}
There are two ways to set up Eclipse with Android.
If Eclipse is installed, follow
the instructions "Installing the Eclipse Plugin" 
at \cite{android} in order to configure the 
Android support for the Eclipse environment.

If Eclipse is not installed yet, Android has released a Bundle that contains Eclipse and all 
necessary tools for developing Android applications.
This bundle can be found at "Get the Android SDK" at \cite{android}.

\subsection{Installing and debugging the application}
The latest version of the frontend code can be found on Github \footnote{Project CS Frontend application. \url{https://github.com/project-cs-2012}}.

In order to run the application on an Android device, connect the device
to a computer, import the project and simply run it in Eclipse. 
Eclipse should recognize the device on its own and immediately offer a list
of available devices, on which the application can be installed on.

For debugging, USB debugging on the device must be enabled.
This setting can be found under the settings menu of the device.

Note that the system might not detect some devices. This issue occurred with
using the HTC ONE x. In that case, a preliminary set up needs to be done.
This set up includes creating a \textit{udev} rules file. More information
can be found under \textbf{"Setting up a Device for Development"} at
\cite{android}.



