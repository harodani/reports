
In this section we will take a look at how to configure
our environment to run our application on an Android device. 
This description includes configuring Eclipse with Android in order to further develop on our 
project. We assume that you already have the Java SDK installed.

For our project we have been working with:
\begin{itemize}
\item Eclipse Indigo Service Release 2
\item Android Version 4.1, API Level 16
\end{itemize}

\section{Configuring Eclipse with Android}
There are two ways to set up Eclipse with Android.
If you already have Eclipse installed, follow
\href{http://developer.android.com/sdk/installing/installing-adt.html}{these}
instructions in order to configure the Android support for your Eclipse environment.

In case you have no Eclipse running yet, Android
lately released a Bundle that contains Eclipse and all 
necessary tools for developing Android applications.
This bundle can be found \href{http://developer.android.com/sdk/index.html}{here}.

\section{Installing and debugging the application}
In order to run the application on your Android device, connect the device
to your computer, import the project and simply run it in Eclipse. 
Eclipse should recognize the device on its own and immediately offer a list
of available devices, on which the application can be installed on.

For debugging, you furthermore need to enable USB debugging on your device,
which can be found under the settings menu of your device.

Note that your system might not detect some devices. We ran into that issue
using the HTC ONE x. In that case, a preliminary set up needs to be done.
This set up includes creating a udev rules file; more information
can be found under \textbf{"Setting up a Device for Development"} 
\href{http://developer.android.com/tools/device.html#setting-up}{here}.



