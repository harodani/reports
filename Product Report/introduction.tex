\chapter{Introduction}

Nowadays the usage of technical devices has become irreplaceable. Strongly coupled to
each and every device is the possibility to access the internet. But with increasing
densification of devices comes network congestion: problems of
browsing the web during a train ride or a concert are apparent. Simply put, the curent
location-based networking was not addressed for today's idea of massive content sharing. 
In order to face these problems Information-Centric Networking (ICN) was introduced.
The idea behind ICN is to shift the focus from hosts that serve a content to the actual objects that
are shared. More specifically, these Named Data Objects (NDO) shall no longer be coupled
to a host that is owning the content, but shall ideally be retrievable from \textit{anywhere}.

This report picks a proof-of-concept implementation of the usage of the Network of Information (NetInf) 
(see \sect{netinf}), one out of four concepts that realize ICN. The software was
developed in the context of the 
\textit{Project Computer Science}\footnote{\url{http://www.it.uu.se/edu/course/homepage/projektDV/ht12}},
a course under supervision of Olle G\¨{a}llmo and in cooperation with Ericsson Research\footnote{\url{http://www.ericsson.com/}}
in 2012/2013. The goals set for this project are listed in \sect{goals}. 
The product itself consists of two separate parts. One is an Erlang implementation of a Name Resolution Service (NRS) with streaming capabilities
as well as a browser application for Android phones that takes advantage of NetInf services. 
Both products are described more in detail in \sect{product}. For a thorough understanding
of the implementation and extension of the current state, preliminaries and system architecture decisions are 
described in \sect{preliminaries} and \sect{architecture}. 
The performance of both products are evaluated in \sect{evaluation} and based on the results,
\sect{conclusions} discusses the future work and draws conclusions for the developed product
in focus. Finally, the appendix contains installation as well as maintenance instructions.