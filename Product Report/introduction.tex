\chapter{Introduction}
\section{Background}
The current architecture of the Internet is based on a host-to-host based model of communication. Much of the content transferred over the network is generated by a single source and accessed by many recipients. This type of communication is not well-suited for the current Internet architecture \cite{ICNarticle}.

An alternative to host-to-host based networking is Information Centric Networking (ICN). In the ICN model data is requested and fetched regardless of its location. Currently four major specifications of ICN exist \cite{netinf}. 

\section{NetInf enabled applications}
The Network of Information (NetInf) is one of four major existing ICN specifications. NetInf is designed to run independently or on top of current network topologies such as TCP/IP, UDP, Bluetooth etc \cite{netinfproto}.

\subsection{A NetInf based web browser for Android}
When many people in a confined area use 3G-devices to retrieve content simultaneously there is a high load on the common 3G uplink. The \textit{Elephant} web browser is designed to enhance the browsing experience under such conditions by employing an information centric approach to the retrieval of web content. 

\subsection{An Erlang implementation of NetInf}
In order to support the Elephant web browser an implementation of the NetInf specification has been developed.


%With increasing densification of devices comes network congestion: problems of
%browsing the web during a train ride or a concert are apparent. Simply put, the current
%location-based networking is not addressed for today's idea of massive content sharing. 
%In order to face these problems Information-Centric Networking (ICN) was introduced. For
%more information about ICN and all its existing architectures (Data-Oriented Network Architecture (DONA),
%Content-Centric Networking (CCN), Publish-Subscribe Internet Routing Paradigm (PSIRP) and Network of Information (NetInf)) 
%see \cite{netinf}.
%The idea behind ICN is to shift the focus from hosts that serve a content to the actual objects that
%are shared. More specifically, these Named Data Objects (NDO) shall no longer be coupled
%to a host that is owning the content, but shall ideally be retrievable from \textit{anywhere}.

%This report picks a proof-of-concept implementation of the usage of NetInf
%(see \sect{netinf}), one out of four concepts that realize ICN. The software was
%developed in the context of the 
%\textit{Project Computer Science}\footnote{\url{http://www.it.uu.se/edu/course/homepage/projektDV/ht12}},
%a course under the supervision of Olle G\"{a}llmo and in cooperation with Ericsson Research \cite{ericsson}
%in 2012/2013. The goals set for this project are listed in \sect{goals}. 

%The product itself consists of two separate parts developed by two development teams within the same project group. One is an Erlang \cite{erlang} implementation of a Name Resolution Service (NRS) with streaming capabilities.
%The other product is a browser application called "Elephant" for Android \cite{android}  phones that takes advantage of NetInf services that
%are based on OpenNetInf \cite{opennetinf}, an open source Java implementation of NetInf. 

%Both products are described more in detail in \sect{product}. For a thorough understanding
%of the implementation and extension of the current state, preliminaries and system architecture decisions are 
%described in \sect{preliminaries} and \sect{architecture}. 
%The performance of both products are evaluated in \sect{evaluation} and based on the results,
%\sect{conclusions} discusses the future work and draws conclusions for the developed products. Finally, the appendix contains installation as well as maintenance instructions.
