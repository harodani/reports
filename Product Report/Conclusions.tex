\section{Conclusions}
The main goals of this project was to develop applications based on the principles of information centric networking using the NetInf protocol. These goals were achieved and the teams were able to build the applications using Java and Erlang/OTP. The backend product(NetInf NRS) is a concurrent and fault tolerant application as per the principles of OTP. Both teams had clear goals when they started the project and achieved it comfortably in the end. Infact, the backend team added functionality to support streaming videos using the application. This functionality was not part of the original plan but because the team achieved all the other major goals well before time.

Both front-end and back-end teams performed testing and evaluation of the application. The back-end team evaluated the streaming with pure NetInf messages and the modified version of streaming based on the draft. Other results that the back-end team observed during the evaluatuion of streaming was that the list database is very slow. This is because the search time is T*N where T is the number of search tokens and N is the number of meta data attribute stored. But as mentioned in the future work section, some other database should be implemented to see if the application works better. In the modified version of streaming we can improve the polling strategy to transfer all the chunks faster. Also the content validation is disabled in the modified version of video streaming and that can cause unauthorized content to be published. The front-end evaluation showed that apart from two problems the application seems to be working as expected. The first problem is the slow searches, which is mentioned above. The second is the problem with NetInfService randomly pausing, which might result from how the Android OS handles background applications.

 
