\subsection{Backend}

The following section will describe how the NetInf NRS application will be evaluated and tested. One of the goals of the project was to show that the use of the new Network of Information protocol( NetInf) would be better in terms of bandwidth, speed and time compared to existing location based infrastructure. 

The Erlang NetInf NRS application, along with the video stream was evaluated with the following set of tests. 

\begin{description}
\item [Video Streaming protocol evaluation]
The client requested the development team to look into the application of NetInf to real world problems, one of the problems discussed was reducing congestion for broadcasting content(video streaming). 
\begin{itemize}
\item Testing the impure version of NetInf video streaming 
\item Testing the pure version of NetInf video streaming
\item Comparison between both implementations of the NetInf Video streaming
\end{itemize}
\item[NetInf NRS]
The backbone of the entire application, this is the main product which the client requested. It is important to show here just how much better this implementation of NetInf is compared to location based services used today. 

\subsubsection{Notes on Interoperability}

There already are existing implementations done by SAIL and Ericsson Research for the NetInf protocol and in the beginning of the product life-cycle the client requested the development team to try to evaluate the interoperability of this product and others, however as the product evolved the client's requested that the interoperability be left to them to evaluate and for this development team to continue with evaluating the video streaming.

Therefore the development team did not evaluate interoperability at all but there is confidence that with some minor tweaking of the code (due to differences in the various draft versions of the NetInf protocol spec) that this product and others will become interoperable with minimal effort.


\begin{itemize}
\item Evaluation of the search time and get time for the supported databases
\item Resource usage
\item Measuring the number of requests per the frontend phone client
\end{itemize}
\end{description}