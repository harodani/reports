\subsection {Configurations}

Since the system is supposed to be modular, configuration files are also implemented. 

Configuration files allow the user of the system to quickly start it with a predetermined setup, things such as which database to use, which convergence layers are supported, and timers for various functionality are also stored here for quick use and editing. The benifit of using erlang specific configuration files is the great way they are organized, giving a highly readable and easy to swap out functionality. 

We have created 2 config files which live in the root netinf\_nrs directory.

\begin {itemize}
\item list.config
\item riak.config
\end {itemize}

For new databases we encourage a new configuration to be make in order to keep things simple.


The following is the syntax for a config file 

\begin {verbatim}
[{netinf_nrs,
	[
	{key1, value1},
	...
	{keyN, valueN}
	]
}].

\end{verbatim}
And here is an example of how config files look
\begin {verbatim}
[{netinf_nrs,
	[
	{database, nn_database_riak},
	{convergence_layers, ["http"]},
	{ip_timer, 5000},
	{discovery, on},
	{nrs_port, 9999},
	{ct_port, 8078},
	{client_port, 8079}
	]
}].

\end{verbatim}

\subsubsection {Meaning of the config values}
\begin{description}
\item[database]
This is used to define what database to use. The value must be the name of a module that implements the nn\_database behavior. Default is our riak implementation
\item[convergence\_layers]
Deprecated. This was used to define which convergence layers the node should support. Currently udp multicast is used instead
\item[ip\_timer]
Deprecated. This was used to define how often the node broadcasted that it was live to other nodes via udp multicasts
\item[discovery]
Deprecated. This was used to turn of the nodes discovery service for testing purposes
\item[nrs\_port]
This defines what port the NRS will use to listen for NetInf messages
\item[ct\_port]
\underline{\textbf{TODO MARKUS FIX THIS SHIT}}
\item[client\_port]
\underline{\textbf{TODO MARKUS FIX THIS SHIT}}
\end{description}

\subsection {Using config files}

Choosing to run the system with a config file is used by flagging it.

\begin {verbatim}
 erl -pa ebin deps/*/ebin -config configs/list
\end{verbatim}

OR

\begin {verbatim}
 erl -pa ebin deps/*/ebin -config configs/riak
\end{verbatim}

Special note: if the netinf\_nrs.app.src file has some configuration options in the env section and there is a config file specified on the erlang command line then the parameters in the config file  will take precendence.

\subsection {Extracting the config parameters}

Extracting the config parameters is done by running the following command within erlang code or within the erlang shell

\begin {verbatim}
application:get_env(app-name,parameter-name) 
\end{verbatim}

where app-name is the name of the application, in this case netinf\_nrs, and parameter-name is the name of one of the parameters defined in the config file OR the env section of the netinf\_nrs.app.src file.

for example to get the database:

\begin {verbatim}
application:get_env(netinf_nrs,database) 
\end{verbatim}

would return {ok, nn\_database\_list} or {ok, nn\_database\_riak} depending on the configuration. 


\subsubsection  {Erlang config files}

for more information see: http://www.erlang.org/doc/man/config.html
