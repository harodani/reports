\subsection {Configurations}
\label{backconfig}
Since the system is supposed to be modular, configuration files are also implemented. 

Configuration files allow the user of the system to quickly start it with a predetermined setup, things such as which database to use, which convergence layers are supported, and timers for various functionality are also stored here for quick use and editing. The benefit of using Erlang specific configuration files is the great way they are organized, giving a highly readable and easy to swap out functionality. 

We have created three(3) config files which live in the "config" directory.

\begin {itemize}
\item list.config
\item riak.config
\item static\_peers.config
\end {itemize}

Note: the static\_peers configuration is used only for testing of the HTTP forwarding. This configuration contains a list of IP addresses to other NRS'.

For new databases the developers encourage a new configuration to be created in order to keep things simple.

The following is the syntax for a config file 

\begin {verbatim}
[{netinf_nrs,
	[
	{key1, value1},
	...
	{keyN, valueN}
	]
}].

\end{verbatim}
And here is an example of a config file:
\begin {verbatim}
[{netinf_nrs,
	[
	{database, nn_database_riak},
	{convergence_layers, ["http"]},
	{ip_timer, 5000},
	{discovery, on},
	{nrs_port, 9999},
	{ct_port, 8078},
	{client_port, 8079}
	]
}].

\end{verbatim}

\subsubsection {Meaning of the config values}
\begin{description}
\item[database]
This is used to define what database to use. The value must be the name of a module that implements the nn\_database behaviour. Default is the Riak implementation.
\item[convergence\_layers]
Deprecated. This was used to define which convergence layers the node should support. Currently UDP multicast is used instead.
\item[ip\_timer]
Deprecated. This was used to define how often the node broadcasted that it was live to other nodes via UDP multicasts.
\item[discovery]
Deprecated. This was used to control whether or not the discovery service was active for testing purposes.
\item[nrs\_port]
This defines what port the NRS will use to listen for NetInf messages
\item[ct\_port]
This defines the port used to transfer HTTP chunks to clients.
\item[client\_port]
This defines the port the HTML client interface interacts with.
\end{description}

\subsection {Using config files}

To run the application with a config file, the \emph{-config} flag must be set on the Erlang command line.

\begin {verbatim}
 erl -pa ebin deps/*/ebin -config configs/list
\end{verbatim}

OR

\begin {verbatim}
 erl -pa ebin deps/*/ebin -config configs/riak
\end{verbatim}

Note: if the netinf\_nrs.app.src file has some configuration options in the env section and there is a config file specified on the Erlang command line then the parameters in the config file  will take precedence.

\subsection {Extracting the config parameters}

Developers can use the following code to extract the values associated with a configuration parameter.

\begin {verbatim}
application:get_env(app-name, parameter-name). 
\end{verbatim}

The argument app-name is the name of the application, in this case netinf\_nrs, and parameter-name is the name of one of the parameters defined in the config file or the env section of the netinf\_nrs.app.src file.

For example to retrieve the value associated with the database a developer can use the following:

\begin {verbatim}
application:get_env(netinf_nrs, database).
\end{verbatim}

This code would return \{ok, nn\_database\_list\} or \{ok, nn\_database\_riak\} depending on the configuration. However if the parameter name is not defined, the above code will return undefined.

