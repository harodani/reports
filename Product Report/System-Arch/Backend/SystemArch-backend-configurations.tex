\subsection {Configurations}

Since the system is supposed to be modular, we have also implemented configuration files. 

Configuration files allow the user of the system to quickly start it with a predetermined setup, things such as which database to use, which convergence layers are supported, and timers for various functionality are also stored here for quick use and editing. The benifit of using erlang specific configuration files is the great way they are organized, giving you a highly readable and easy to swap out functionality. 

We have created 2 config files which live in the root netinf\_nrs directory.

\begin {itemize}
\item list.config
\item riak.config
\end {itemize}

For new databases we encourage a new configuration to be make in order to keep things simple.


The following is the syntax for a config file 

\begin {verbatim}
[{netinf_nrs, [ 
	     {database, nn_database_list},
	     {convergence_layers, ["http"]},
	     {ip_timer, 5000},
	     {discovery, on} ]}].

\end{verbatim}

\subsection {Using config files}

We now need to have a flag on the erl command line that specifies the name of the config file to use. 

\begin {verbatim}
 erl -pa ebin deps/*/ebin -config configs/list
\end{verbatim}

OR

\begin {verbatim}
 erl -pa ebin deps/*/ebin -config configs/riak
\end{verbatim}

Special note: if the netinf\_nrs.app.src file has some configuration options in the env section and there is a config file specified on the erlang command line then the parameters in the config file  will take precendence.

\subsection {Extracting the config parameters}

In the erlang code we use the following 

\begin {verbatim}
application:get_env(app-name,parameter-name) 
\end{verbatim}

where app-name is our name for netinf\_nrs and parameter-name is the name of one of the parameters defined in the config file OR the env section of the netinf\_nrs.app.src file.

for example to set the database:

\begin {verbatim}
application:get_env(netinf_nrs,database) 
\end{verbatim}

would return {ok, nn\_database\_list} or {ok, nn\_database\_riak} depending on the configuration. 


\subsubsection  {Erlang config files}

for more information see: http://www.erlang.org/doc/man/config.html
