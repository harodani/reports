\subsection{Erlang}
The backend group decided with the client to have our product written in Erlang because of it's fault-tolerance, scalability and ease of distribution. Erlang is also a product of research done by the client in the past. This means that we have access to many individuals who are competent in the language at our disposal.
Erlang uses the idea of modules and nodes as a primary platform for serving a function, this allows our product to be broken up into several parts.
\subsection{Javascript}

Javascript was used when the backend group decided to create a simple http interface to the NetInf Name Resolution server in order to show a proof of concept (NetInf streaming). Javascript was used to calculate the hash of files for streaming.