\subsection{Erlang}
Erlang is a concurrent, functional, fault-tolerant language with great scalability and ease of distribution. It was developed by Ericsson in the mid 80's and became open source 1998.\cite{otpInAction} These factors among others such as the client being Ericsson Research made Erlang the choice of language for the NRS implementation.
Another reason for choosing Erlang is that it uses the idea of modules and nodes as a primary platform for serving a function. This allowed the product to be broken up into several parts, supporting concurrent development.
\subsection{Javascript}
Javascript was used when the backend development team decided to create a simple HTTP interface to the NetInf Name Resolution Service in order to show a proof of concept (NetInf streaming). Javascript was used to calculate the hash of files for streaming as well as for asynchronous communication with the NRS.
