\subsection{Information-centric Networking}
In the dawn of internet it was designed based on a host-to-host architecture, so to say a one-to-one communication architecture. This worked well for the time being, the most popular services at the time was e-mailing. As the amount of users of the web has increased over the years so has the amount of services used. Today most of the traffic on the web involves transfering of media or social networking, both relying on one-to-many communication. The internet that we use today was not designed for these kinds of applications and services. 
Even though the continous improvements over the years, the TCP/IP paradigm is not suitable for one-to-many communication. 
Today it's easier to send a document to a friend across the globe than it is to share that very same document to a group of people sitting in the same room as you are using local network resources\cite{ICNarticle}.
This is where Information-centric Networking(ICN) comes in. ICN is a new way of thinking, a new way of transfering data. ICN is all about removing the need of knowing where to get the data and instead focuses on how to get the data\cite{ICNarticle}, it strives to remove the dependencies of one-to-many connections and instead use many-to-many communication.

\subsection{Network of Information}
Network of Information(NetInf) is one of the first approaches proposed by the 4WARD project. \cite{4ward} This ICN paradigm was intended to deal with the issues that the current host-centric networks suffers from. Every object on the network is called an Named Data Object(NDO) and is self-verifying, this leads to the user being able to request a certain object, an NDO, and fetch it from any source without worrying who or where it gets it from. The NDO's verifying itself by using its own hash value as part of its name along with the used hash-algorithm.
A lot has changed in NetInf since the 4WARD project made the first draft, currently the most recent versions are managed by the SAIL project. See Appendix XXX.

\subsection{OpenNetInf}
OpenNetInf \cite{opennetinf} is an open source implementation of NetInf in Java developed at the University of Paderborn. OpenNetInf is still in the very early development phase and mainly aimed at research. The frontend group decided to use OpenNetInf as the starting point for their NetInf functionality, but still had to implement and extend it to provide the needed functionality. One reason for choosing OpenNetInf, rather than starting from scratch, was to not have to reinvent the wheel. It was also a chance to contribute to an existing project. Another reason was the closely related work done in a previous master thesis \cite{masterthesis} which used OpenNetInf.
