\subsection{Information-centric Networking}
\label{sec:netinf}
The Internet was originally designed based on a host-centric paradigm (one-to-one communication), where users explicitly connect to hosts in order to use services and retrieve resources. In the early days this worked well due to the low amount of users per host, but as the internet gained popularity some services began to be used increasingly. Over the past decades, the host-centric approach has become a growing impediment for services with large user bases, with workarounds like load-balancing and content delivery networks to circumvent bandwidth bottle necks in place. Today most traffic involves transferring of audio/video media and social networking content, both relying on one-to-many communication. Information centric networking (ICN) is a research field aiming to redesign the internet in a fundamental way for today's and the near future usage patterns. In ICN, the actual host providing a specific resource or service can be arbitrary and therefore unknown to the user. Instead of connecting to a host, the user queries the network as a whole. This enables low-level caching in every network node, so that repeated forwarding of identical information can be minimised and bandwidth be used more efficiently. The main challenges in ICN are the ways of addressing information units and integration with existing, host-centric networks. At this time ICN only exists in the form of independent research projects (e.g. NetInf), with no cross-industry standards on the horizon yet \cite{ICNarticle}. 


\subsection{Network of Information}
Network of Information(NetInf) is one of the first approaches proposed by the 4WARD project. \cite{4ward} This ICN paradigm was intended to deal with the issues that the current Host-Centric Networks suffer from. Every object on the network is called an Named Data Object(NDO) and is self-verifying, this leads to the user being able to request a certain object, an NDO, and fetch it from any source without worrying who or where it gets it from. The NDO can verify itself by using its own hash value as part of its name along with the used hash-algorithm.
A lot has changed in NetInf since the 4WARD project made the first draft. Currently the most recent versions are managed by the SAIL project. \cite{netinfproto}

\subsection{OpenNetInf}
OpenNetInf \cite{opennetinf} is an open source Java implementation of NetInf developed at the University of Paderborn. OpenNetInf is still in the very early development phase and mainly aimed at research. The frontend development team decided to use OpenNetInf as the starting point for the android client's NetInf functionality, but still had to implement and extend it to provide the needed functionality. One reason for choosing OpenNetInf, rather than starting from scratch, was not having to reinvent the wheel. It was also a chance to contribute to an existing project. Another reason was the closely related work done in a previous master thesis \cite{masterthesis} which used OpenNetInf.
