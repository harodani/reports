\section{Backend}
This section describes the goals and scope set by the backend team for this product.

\subsection{Goals}
The following goals were set for the Erlang implementation of the Name Resolution Service (NRS):
\begin{enumerate}
 \item {Build an NRS for the Erlang NetInf application.}\\
 \item {Be able to publish, store and retrieve Named Data Objects (NDO's) in a NetInf network.}\\
 \item {Make each NetInf node a caching node that can store NDO's.}\\
 \item {Make the back-end processes distributed, concurrent and fault tolerant.}\\ 
 \item {Be able to stream video using our Erlang NetInf application.}\\
  \end{enumerate}

\subsection{Scope}
The scope of the NRS application is limited to providing all the functionalities outlined in the NetInf Protocol draft document. \cite{netinfproto} This document outlines what information a NetInf Get, Publish and Search message should contain. It also defines how a Get, Publish and Search response message should look like. Apart from that it also covers specifications for the HTTP and UDP convergence layers. We made sure that our application followed all these specifications accurately. Providing video streaming was not part of the scope at the beginning of this project but at the client's request preliminary(proof-of-concept) work was done, however readers should note that the video streaming is not meant to be a complete product and the development team encourages further research into this product.
