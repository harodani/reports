\section{Goals and Scope}

\section{Frontend}

\subsection{NetInf Enabled Browser}

The goal of the frontend is to produce a NetInf enabled browser for Android devices. The browser utilizes NetInf technology to retrieve web pages from other nearby devices and/or caching nodes in order to reduce the usage of shared 3G uplinks. The NetInf messages should conform to the NetInf HTTP convergence layer \cite{netinfproto}. Web pages are split into several parts to make them available from multiple sources.

NetInf protocol mandates that the browser have the possibility to:

\begin{itemize}
	\item Inject existing web pages into the NetInf network as NDOs.
	\item Given a traditional web URL be able to find the corresponding NDO.
	\item Get NDOs from other devices.
\end{itemize}

The following problems are considered to be out of scope:

\begin{itemize}
	\item Privacy
	\item Security
	\item Dynamic Content
	\item Battery Consumption
	\item Bluetooth Congestion
\end{itemize}

Privacy and security are both very important aspects, but would require complex considerations. They are areas that require future work.

Dynamic content is relevant since a lot of content that is of interest to many users is dynamic and changes often (e.g. newspapers, Facebook, Twitter). However this adds a lot of complexity to the problem since dynamic content means the mappings from traditional web URLs to NDOs are constantly changing.

Battery consumption is a serious issue in this type of application. The application uses Bluetooth which is a battery draining technology. The simplest solution here is to let the user disable Bluetooth. Hopefully the problem of heavy battery consumption will be solved by future technological advancements.

A final problem that is not taken into consideration is possible congestion in the Bluetooth network due to a large amount of devices running simultaneously.
