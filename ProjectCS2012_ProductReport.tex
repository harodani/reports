\documentclass[11pt]{report}
%Gummi|061|=)
\title{Project CS 2012 Product Report\\Uppsala University\\}

\author{Daniele Bacarella\\
Jon Borglund\\
Kiril Goguev\\
		Faroogh Hassan\\
		Marcus Ihlar\\
		Alexander Lindholm\\
		Knut Lorenzen\\
		Thomas Nordström\\
}

\date{}
\begin{document}

\maketitle

\tableofcontents

\chapter{Introduction}
\chapter{Information Centric Networking}

\subsection{Network of Information}
\subsection{Problems  today}
\section{Development Languages}
\subsection{Java-Android}
\subsection{Erlang}
The backend group decided with the client to have our product written in Erlang because of it's fault-tolerance, scaleability and ease of distribution. Erlang is also a product of research done by the client in the past. This means that we have access to many indivudals who are competent in the language at our disposal. 
\subsection{Javascript}

Javascript was used when the backend group decided to create a simple http interface to the Netinf Name Resolution server in order to show a proof of concept (NetInf streaming). Javascript was used to calculate the hash of files for streaming.

\subsection{Testing}
\chapter{Product Description}
\section {Frontend - Android Client}
\section {Backend - Erlang NetInf Name Resolution Server}
\subsection{NetInf Video Streaming}
\chapter{System Architecture}
\section {Android Client}
\section {Netinf NRS}
\chapter{Evaluation and testing}
\chapter{Related work}
\chapter{Future Work}
\chapter{Conclusion}
\chapter{Installation Instructions}
\section {Frontend}
\subsection {Dependencies}
\section {Backend}
\subsection {Dependencies}
\end{document}
